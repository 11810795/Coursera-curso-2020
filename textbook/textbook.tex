\PassOptionsToPackage{table}{xcolor}
\documentclass[justified]{tufte-book}
\usepackage{mooculus}

\usepackage{mathtools}
\DeclarePairedDelimiter{\ceil}{\lceil}{\rceil}
\DeclarePairedDelimiter{\floor}{\lfloor}{\rfloor}

%\usepackage{showkeys} %% Useful for debugging

\setcounter{secnumdepth}{2}
\usepackage{ulem}
\usepackage{amssymb}
\newcommand{\N}{\mathbb{N}}
\newcommand{\Z}{\mathbb{Z}}


% Prints the month name (e.g., January) and the year (e.g., 2008)
\newcommand{\monthyear}{%
  \ifcase\month\or January\or February\or March\or April\or May\or June\or
  July\or August\or September\or October\or November\or
  December\fi\space\number\year
}

% Generates the index
\usepackage{makeidx}
\makeindex

\newcommand{\xrefn}[1]{\ref{#1}}



\newenvironment{lemma}{\subsection*{Lemma}}{}
\newenvironment{remark}[1]{\subsection*{Remark: #1}}{}


%\def\exam{\null}
\def\pagerdef{\null}

\def\dfont{\bf}
\def\em{\it}           % for emphasis

\newcommand{\ds}{\displaystyle}

\let\ssk\smallskip \let\msk\medskip \let\bsk\bigskip

\usepackage{multicol}
\def\twocol{\begin{multicols}{2}}
\def\endtwocol{\end{multicols}}

\title{Sequences and Series}
%\author{Jim Fowler and Bart Snapp}
\publisher{This document was typeset on \today.}
%\newcommand{\blankpage}{\newpage\hbox{}\thispagestyle{empty}\newpage}

%% % Prints an epigraph and speaker in sans serif, all-caps type.
%% \newcommand{\openepigraph}[2]{%
%%   %\sffamily\fontsize{14}{16}\selectfont
%%   \begin{fullwidth}
%%   \sffamily\large
%%   \begin{doublespace}
%%   \noindent\allcaps{#1}\\% epigraph
%%   \noindent\allcaps{#2}% author
%%   \end{doublespace}
%%   \end{fullwidth}
%% }



\begin{document}
\frontmatter

\maketitle

\pagebreak

% v.4 copyright page

\begin{fullwidth}
~\vfill
\thispagestyle{empty}
\setlength{\parindent}{0pt}
\setlength{\parskip}{\baselineskip}
Copyright \copyright\ \the\year\ Jim Fowler and Bart Snapp

This work is licensed under the Creative Commons
Attribution-NonCommercial-ShareAlike License. To view a copy of this
license, visit
\url{http://creativecommons.org/licenses/by-nc-sa/3.0/}~or send a~or send a
letter to Creative Commons, 543 Howard Street, 5th Floor, San
Francisco, California, 94105, USA. If you distribute this work or a
derivative, include the history of the document. 

The source code is available
at: \url{https://github.com/ASCTech/mooculus/tree/master/public/textbook}

\noindent
This text is based on David Guichard's open-source calculus text which
in turn is a modification and expansion of notes written by Neal
Koblitz at the University of Washington. David Guichard's text is
available at \url{http://www.whitman.edu/mathematics/calculus/}~under a Creative Commons license.

\noindent The book includes some exercises and examples from {\it
  Elementary Calculus: An Approach Using Infinitesimals}, by H.~Jerome
Keisler, available at
\url{http://www.math.wisc.edu/~keisler/calc.html}~under a Creative
Commons license. In addition, the chapter on differential equations
and the section on numerical integration are largely derived from the
corresponding portions of Keisler's book.  Albert Schueller, Barry
Balof, and Mike Wills have contributed additional material.

\noindent This book is typeset in the Kerkis font, 
Kerkis \copyright~Department of Mathematics, University of the Aegean.


\noindent We will be glad to receive corrections and suggestions for
improvement at \texttt{fowler@math.osu.edu} or
\texttt{snapp@math.osu.edu}.

\end{fullwidth}

\tableofcontents


%% \renewcommand{\listtheoremname}{List of Main Theorems}
%% \setcounter{tocdepth}{1}
%% \listoftheorems[numwidth=4em,ignoreall,show={mainTheorem}]



%\chapter*{List of Main Theorems}% uses ntheorem
%\theoremlisttype{opt} 
%\listtheorems{mainTheorem}


%\listoffigures

%\listoftables

% r.7 dedication
%\cleardoublepage
%~\vfill
%\begin{doublespace}
%\thispagestyle{empty}
%\noindent\fontsize{18}{22}\selectfont\itshape
%\nohyphenation
%\centerline{\it For Kathleen,\/}
%\centerline{\it without whose encouragement\/}
%\centerline{\it this book would not have\/}
%\centerline{\it been written.\/}
%\end{doublespace}
%\vfill
%\vfill

% r.9 introduction
%\cleardoublepage

\chapter*{How to \sout{read} do mathematics}

Reading mathematics is \textbf{not} the same as reading a novel---it's
more fun, and more interactive!  To read mathematics you need
\begin{enumerate}
\item a pen,
\item plenty of blank paper, and
\item the courage to write down everything---even ``obvious'' things.
\end{enumerate}
As you read a math book, you work along with me, the author, trying to
anticipate my next thoughts, repeating many of the same calculations I
did to write this book.  You must \textbf{write} down each expression,
\textbf{sketch} each graph, and constantly \textbf{think} about what
you are doing.  You should work examples.  You should fill-in the
details I left out.  This is not an easy task; it is \textbf{hard}
work, but, work that is, I very much hope, rewarding in the end.

Mathematics is not a passive endeavor.  I may call you a ``reader''
but you are not reading; you are writing this book for yourself.

\vspace{0.5in}
\hfill---the so-called ``author''

% Start the main matter (normal chapters)
\mainmatter
%\tikzexternaldisable

%%%%%%%%%%%%%%%%%%%%%%%%%%%%%%%%%%%%%%%%%%%%%%%%%%%%%%%%%%%%%%%%
\chapter*{Introduction, or\ldots what is this all about?}

\input introduction

%%%%%%%%%%%%%%%%%%%%%%%%%%%%%%%%%%%%%%%%%%%%%%%%%%%%%%%%%%%%%%%%
\chapter{Sequences}
\label{chapter:sequences}

%\begin{outline}
%\noindent\startcontents[section]% kind
%\printcontents[section]{}{1}{}% kind, prefix, top, init-code
%\end{outline}

\input sequences

%%%%%%%%%%%%%%%%%%%%%%%%%%%%%%%%%%%%%%%%%%%%%%%%%%%%%%%%%%%%%%%%
\chapter{Series}
\label{chapter:series}

%\begin{outline}
%\noindent\startcontents[section]% kind
%\printcontents[section]{}{1}{}% kind, prefix, top, init-code
%\end{outline}
 
\input series

%%%%%%%%%%%%%%%%%%%%%%%%%%%%%%%%%%%%%%%%%%%%%%%%%%%%%%%%%%%%%%%%
\chapter{Convergence tests}
\label{chapter:convergence-tests}

%\begin{outline}
%\startcontents[convergence-tests]% kind
%\printcontents[convergence-tests]{}{1}{}% kind, prefix, top, init-code
%\end{outline}

It is generally quite difficult---indeed, often impossible---to
determine the value of a series exactly.  Even if we can't compute the
value of a series, in many cases it is possible to determine whether
or not the series converges. We will spend most of our time on this
problem.

\section{Comparison Tests}
\input comparison-test

\section{Ratio tests}
\label{section:ratio-test}
\input ratio-test

\section{Integral test}
\input integral-test

%%%%%%%%%%%%%%%%%%%%%%%%%%%%%%%%%%%%%%%%%%%%%%%%%%%%%%%%%%%%%%%%
\chapter{Alternating series}
\label{chapter:alternating-series}

%\startcontents[sections]% kind
%\printcontents[sections]{p}{1}{}% kind, prefix, top, init-code
%\end{contents}

\section{Alternating series test}
%\input alternating-series

\section{Absolute convergence}
\label{section:absolute-convergence}
%\input absolute-convergence

%%%%%%%%%%%%%%%%%%%%%%%%%%%%%%%%%%%%%%%%%%%%%%%%%%%%%%%%%%%%%%%%
\chapter{Power series}
\label{chapter:power-series}

\startcontents[sections]% kind
\printcontents[sections]{p}{1}{}% kind, prefix, top, init-code
%\end{contents}

\section{Definitions}
%\input power-series
\section{Calculus with power series}
%\input term-by-term

%%%%%%%%%%%%%%%%%%%%%%%%%%%%%%%%%%%%%%%%%%%%%%%%%%%%%%%%%%%%%%%%
\chapter{Taylor series}
\label{chapter:taylor-series}

\startcontents[sections]% kind
\printcontents[sections]{p}{1}{}% kind, prefix, top, init-code
%\end{contents}

\section{Finding Taylor series}
%\input taylor-series

\section{Taylor's Theorem}
%\input remainders

%%%%%%%%%%%%%%%%%%%%%%%%%%%%%%%%%%%%%%%%%%%%%%%%%%%%%%%%%%%%%%%%
\chapter*{Review}
%\input review

%\bibliography{sample-handout}
%\bibliographystyle{plainnat}

\finalizeanswers
\chapter*{Answers to Exercises}
\small
\addcontentsline{toc}{chapter}{Answers to Exercises}
\IfFileExists{answers.tex}{\subsection *{Answers for 1.8}
\hypertarget {a:1.8.1}{\hyperlink {e:1.8.1}{\bfseries 1.}} \mdseries $1$\qquad 
\hypertarget {a:1.8.3}{\hyperlink {e:1.8.3}{\bfseries 3.}} \mdseries $0$\qquad 
\hypertarget {a:1.8.4}{\hyperlink {e:1.8.4}{\bfseries 4.}} \mdseries $1$\qquad 
\hypertarget {a:1.8.5}{\hyperlink {e:1.8.5}{\bfseries 5.}} \mdseries $1$\qquad 
\hypertarget {a:1.8.6}{\hyperlink {e:1.8.6}{\bfseries 6.}} \mdseries $0$\qquad 
}
\normalsize
\backmatter

\addcontentsline{toc}{chapter}{Index}
\printindex


\end{document}


