\nobreak Does the series $\ds\sum_{n=0}^\infty {n^5\over 5^n}$
converge? It is possible, but a bit unpleasant, to approach this with
the comparison test.

\marginnote{Mathematics is more than just about getting answers; a
  goal of mathematics is not only to find truth, but to package the
  resulting ``truth'' in a format that permits another human being to
  understand the reasons for its being true.  Arguments like
  this---which are perfectly convincing but seem entirely
  unmotivated---are, arguably, missing the point.  What has been
  gained if we find something is true but the reason for its being
  true remains inscrutable?}

\begin{example}
  \label{example:n-to-fifth-over-five-to-n}
  The series $\ds\sum_{n=0}^\infty {n^5\over 5^n}$ converges.
\end{example}

\marginnote{This idea---that the first handful of terms do not affect convergence at all---will be discussed formally in Section~\xrefn{section:convergence-for-tails}.}

\begin{solution}
  As long as $n \geq 23$, we have $2^n \geq n^5$.  Therefore, as long as $n \geq 23$, we have
  $$
  \frac{n^5}{5^n} \leq \frac{2^n}{5^n} = \left( \frac{2}{5} \right)^n
  $$
  But the geometric series $\ds\sum_{n=23}^\infty \left( \frac{2}{5}
  \right)^n$ converges, since the ratio between subsequent terms is
  less than one.  So by comparison, the smaller series
  $$
  \ds\sum_{n=23}^\infty \frac{n^5}{5^n}
  $$
  must also converge.  The first handful of terms when $n < 23$
  doesn't affect convergence at all, so we are justified to conclude
  that the original series $\ds\sum_{n=0}^\infty {n^5\over 5^n}$
  converges.
\end{solution}

That worked---but it invoked an unmotivated fact: how do I know that
$2^n \geq n^5$ whenever $n \geq 23$?  Invoking that fact seems a bit
random---yes, yes, a proof, but perhaps not a proof that conveys
exactly what is going on.  It is a valid argument, but missing some
motivation.

\subsection{Theory}
\label{subsection:theory-of-ratio-test}

Instead, consider what happens as we move from one term to the next term in
this series, that is, consider two neighboring terms
$$\cdots+{n^5\over5^n}+{(n+1)^5\over 5^{n+1}}+\cdots.$$
The denominator goes up by a factor of 5, $\ds 5^{n+1}=5\cdot5^n$, but
the numerator goes up by much less: $\ds
(n+1)^5=n^5+5n^4+10n^3+10n^2+5n+1$, which is much less than $\ds 5n^5$
when $n$ is large, because $\ds 5n^4$ is much less than $\ds n^5$.
(This sort of thinking is \textit{why} it was worth comparing $n^5$ to
an exponential $2^n$ for $n$ large.)

So we might guess that in the long run---when $n$ is very large---it
begins to look as if each term is about $1/5$ of the previous term. We
have seen series that behave like this, namely the geometric series
$$\sum_{n=0}^\infty {1\over 5^n} = {5\over4}.$$
We are beginning to see why it made sense to compare the given series
to a geometric series as in the initially very unmotivated argument
above.

But we can do better!  Instead of an \textit{ad hoc} argument which
compared $n^5$ to $2^n$, we can try to make rigorous the idea that
each term is ``eventually'' about a fifth as big as the previous term.
The key is to notice that
$$
  \lim_{n\to\infty} {a_{n+1}\over a_n}=
  \lim_{n\to\infty} {(n+1)^5\over 5^{n+1}}{5^n\over n^5}=
  \lim_{n\to\infty} {(n+1)^5\over n^5}{1\over 5}=1\cdot {1\over5}
    ={1\over 5}.
$$ 
This is a more formal version of what we noticed about the ratio of
subsequent terms: in the long run, each term is one fifth of the
previous term.  Pick some number just slightly bigger than
$\ds\frac{1}{5}$; let's call that number $\ds\frac{1}{5} + \epsilon$.
\marginnote{The symbol $\epsilon$ is the Greek letter
  \textit{epsilon}, and, conventionally, denotes a small but positive
 number.}

% BADBAD: I suspect the switch from n to N is pretty confusing, so
% this could be rewritten to be less confusing.

Because
$$\lim_{n\to\infty} {a_{n+1}\over a_n}={1\over5},$$
then by choosing $n$ big enough, say $n\ge N$ for some $N$, we can
guarantee that $\ds\frac{a_{n+1}}{a_n}$ is as close as we'd like to
$\ds\frac{1}{5}$, say within $\epsilon$ of $\ds\frac{1}{5}$.  More
succinctly, there must be some $N$ so that whenever $n \geq N$, we
have
$$
\frac{a_{n+1}}{a_n} < \frac{1}{5} + \epsilon
$$
Multiplying both sides by $a_n$, we find whenever $n \geq N$ that
$$
a_{n+1} <  \left( \frac{1}{5} + \epsilon \right) a_n.
$$
We can say the same thing when $n$ is replaced by $N+1$, meaning
$$
a_{N+2} <  \left( \frac{1}{5} + \epsilon \right) a_{N+1},
$$
which together with the previous statement when $n$ is $N$ implies
$$
a_{N+2} <  \left( \frac{1}{5} + \epsilon \right)^2 a_N.
$$
And we can repeat this again!  Since
$$
a_{N+3} <  \left( \frac{1}{5} + \epsilon \right) a_{N+2},
$$
we also learn that
$$
a_{N+3} <  \left( \frac{1}{5} + \epsilon \right)^3 a_N,
$$
or in general,
$$
a_{N+k} <  \left( \frac{1}{5} + \epsilon \right)^k a_N.
$$
This is setting up a comparison.  The geometric series
$$
\sum_{k=0}^\infty \left( \frac{1}{5} + \epsilon \right)^k a_N = \left( \sum_{k=0}^\infty \left( \frac{1}{5} + \epsilon \right)^k \right) a_N
$$
converges as long as $\epsilon$ had been chosen so small that $\ds\frac{1}{5} + \epsilon < 1$.  But if that series converges, then the series 
$$
\sum_{k=0}^\infty a_{N+k} = a_N + a_{N+1} + a_{N+2} + \cdots
$$
also converges, because each term of that series is smaller than the corresponding term of the convergent geometric series.  But if \textit{that} series converges, then 
$$
\sum_{k=0}^\infty a_{k} = a_0 + a_{1} + a_{2} + \cdots
$$
converges since we've just added the number $\ds a_0+a_1+\cdots+a_{N-1}$ to the convergent series $\sum_{k=0}^\infty a_{N+k}$.

Under what circumstances could we do this? What was crucial was that
the limit of $\ds a_{n+1}/a_n$, say $L$, was less than 1 so that we
could pick a value $\epsilon$ so that $L + \epsilon < 1$ and then
compare to the convergent series
$$
\sum_{k=0}^\infty (L + \epsilon)^k \cdot a_N.
$$
That's really all that is required to make the argument work. We also
made use of the fact that the terms of the series were positive.
\marginnote{In general, we can consider instead the absolute values of
  the terms, and end up testing for \textbf{absolute convergence.}
  We'll discuss this topic in
  Section~\xrefn{section:absolute-convergence}.}  Let's summarize the
situation when this works.

\begin{theorem} (The Ratio Test)
  \label{theorem:ratio-test}
Consider the series $\ds\sum_{n=0}^\infty a_n$ where each term $a_n$ is positive.
Suppose that
$$
\lim_{n\to \infty} \frac{a_{n+1}}{a_n}=L.
$$
If $L<1$ the series $\ds\sum_{n=0}^\infty a_n$ converges.  If $L>1$ the
series diverges.  If $L=1$ this test is inconclusive.
\end{theorem}

\begin{proof}
  Example~\xrefn{example:n-to-fifth-over-five-to-n} essentially proves
  the first part of this, if we simply replace $1/5$ by $L$ and $1/2$
  by $r$.  So when $L< 1$, the series converges.

Let's consider the other situation.  Suppose that $L>1$, and pick $r$
so that $1<r<L$.  Then for $n\ge N$, for some $N$,
$${|a_{n+1}|\over |a_n|} > r \hbox{\quad and\quad} |a_{n+1}| > r|a_n|.$$
This implies that $\ds |a_{N+k}|>r^k|a_N|$, but since $r>1$ this means
that $\ds\lim_{k\to\infty}|a_{N+k}| \neq 0$, which means also that
$\ds\lim_{n\to\infty}a_n \neq 0$. By the divergence test, the series
diverges.

Finally, when $L = 1$, the test truly is inconclusive---not because we
aren't clever enough, but because there are both convergent and
divergent series with $L = 1$.  For example, the $p$-series
$\sum_{n=1}^\infty 1/n^2$ converges, and the harmonic series
$\sum_{n=1}^\infty 1/n$ diverges---but in both cases, $L = 1$.
\end{proof}

\subsection{Practice}

The ratio test is particularly useful for series involving
the factorial function.

\marginnote{``It works in practice---but does it work in theory?''
  The ratio test is awfully useful in practice, but I think the theory
  behind it---that we're taking a limit of the ratio between
  neighboring terms in order to compare with a geometric series---is
  quite lovely.}

\begin{example} 
\label{example:ratio-test-factorials}
Does the series 
$$
\ds\sum_{n=0}^\infty  5^n/n!
$$
converge or diverge?
\end{example}

\marginnote{Remember our convention that $0! = 1$.  Despite appearances, we are not dividing by zero.}

\begin{solution}
Let's name the terms $a_n = 5^n / n!$ so we are considering $\sum_{n=0}^\infty a_n$.  To apply the ratio test, we consider
\begin{align*}
  \lim_{n\to\infty} \frac{a_{n+1}}{a_n}
  &= \lim_{n\to\infty} {5^{n+1}\over (n+1)!}{n!\over 5^n} \\
  &= \lim_{n\to\infty} {5^{n+1}\over 5^n}{n!\over (n+1)!} \\
  &= \lim_{n\to\infty} {5}{1\over (n+1)}=0.
\end{align*}
Since $0<1$, the series converges by the ratio test.
\end{solution}

\begin{exercises}

\begin{exercise} Compute $\ds\lim_{n\to\infty} |a_{n+1}/a_n|$ for the series
$\ds\sum_{n=1}^\infty {1\over n^2}$.
\begin{answer} 1
\end{answer}\end{exercise}

\begin{exercise} Compute $\ds\lim_{n\to\infty} |a_{n+1}/a_n|$ for the series
$\ds\sum_{n=1}^\infty {1\over n}$.
\begin{answer} 1
\end{answer}\end{exercise}

Determine whether each of the following series converges or diverges.

\begin{exercise} $\ds\sum_{n=0}^\infty (-1)^{n}{3^n\over 5^n}$
\begin{answer} converges
\end{answer}\end{exercise}

\begin{exercise} $\ds\sum_{n=1}^\infty {n!\over n^n}$
\begin{answer} converges
\end{answer}\end{exercise}

\begin{exercise} $\ds\sum_{n=1}^\infty {n^5\over n^n}$
\begin{answer} converges
\end{answer}\end{exercise}

\begin{exercise} $\ds\sum_{n=1}^\infty {(n!)^2\over n^n}$
\begin{answer} diverges
\end{answer}\end{exercise}

\end{exercises}
