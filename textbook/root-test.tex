There is another convergence test called the \defnword{root test},
which can be justified with an argument not so
different\sidenote{Indeed, justifying the root test makes a good
  exercise for you, the reader, so it is included among the exercises
  for this section.} from that which justified the ratio test in
Subsection~\xrefn{subsection:theory-of-ratio-test}.  The ratio test is
\textit{very occasionally} easier to apply, but usually not as good as
choosing to use the ratio test.

With those disparaging remarks out of the way, let us now state the
root test.

\begin{theorem}[The Root Test] \label{thm:root test}
Consider the series $\ds\sum_{n=0}^\infty a_n$ where each term $a_n$ is positive.
Suppose that $\ds\lim_{n\to \infty} \left(a_n\right)^{1/n}=L$.  Then,
\begin{itemize}
\item if $L<1$ the series $\sum a_n$ converges, 
\item if $L>1$ the series diverges, and
\item if $L=1$, then the root test is inconclusive.
\end{itemize}
\end{theorem}

Let's apply the root test to analyze the convergence of a series.

\begin{example} Analyze $\ds\sum_{n=0}^\infty {5^n\over n^n}$.
\end{example}

\begin{solution}
  Usually, the ratio test is a good choice when the series involves
  $n^{\mbox{\scriptsize th}}$ powers; in this case, the ratio test
  turns out to be a bit difficult on this series, since we have to
  calculate
  $$
  \lim_{n \to \infty} \frac{5^{n+1} / (n+1)^{n+1}}{5^n / n^n}
  $$
  and that may not be entirely obvious.  So we, bedgrudgingly, apply
  the root test, which asks us to calculate
\begin{align*}
  L &= \lim_{n\to\infty} \left({5^n\over n^n}\right)^{1/n} \\
  &= \lim_{n\to\infty} {(5^n)^{1/n}\over (n^n)^{1/n}} \\
  &= \lim_{n\to\infty} {5\over n}=0.
\end{align*}
  Since $L = 0<1$, we may conclude that the given series converges.
\end{solution}

The root test is frequently useful when $n$ appears as an exponent in
the general term of the series---though the ratio test is also useful
in that case.  Technically, whenever the ratio test is conclusive
(i.e., whenever $\lim_{n\to\infty} a_{n+1} / a_n = L \neq 1$), so is
the root test---but not vice versa.  In other words, the root test
\textit{does} work on some series that the ratio test fails on.

\begin{example}
Find a series for which the ratio test is inconclusive, but the root test determines that the series converges.
\end{example}

\begin{solution}
Here is such a situation.  Try using the ratio test on
$\sum_{n=1}^\infty a_n$ where $(a_n)$ is a sequence which ``stutters''
like
$$
a_n = \begin{cases}
1/2^{n/2} & \mbox{if $n$ is even, and} \\
1/2^{(n+1)/2} & \mbox{if $n$ is odd.} 
\end{cases}
$$
Note that $a_1 = a_2$ and $a_3 = a_4$ and $a_5 = a_6$, so
$a_{n+1}/a_n$ is often~1, which messes up the ratio test---indeed,
$\lim_{n \to \infty} a_{n+1}/a_n$ does not exist in this case.
Nevertheless,
$$
\sqrt[n]{a_n} = \begin{cases}
1/2^{1/2} & \mbox{if $n$ is even, and} \\
1/2^{(n+1)/(2n)} & \mbox{if $n$ is odd,} 
\end{cases}
$$
and so $\lim_{n \to\infty} \sqrt[n]{a_n} = 1/\sqrt{2} < 1$, which
means the sequence converges by the root test.  Admittedly, we didn't
need the root test: this series is just a geometric series where the
terms repeat, so it definitely converges.  Still, it proves the point
that the ratio test can fail while the root test succeeds.
\end{solution}

\begin{exercises}
\begin{exercise}
Prove theorem \xrefn{thm:root test}, the root test.\end{exercise}

\begin{exercise} Compute $\ds\lim_{n\to\infty} |a_n|^{1/n}$ for the series
$\sum 1/n^2$.
\end{exercise}

\begin{exercise} Compute $\ds\lim_{n\to\infty} |a_n|^{1/n}$ for the series
$\sum 1/n$.
\end{exercise}

\end{exercises}
