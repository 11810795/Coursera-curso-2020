
\section{Notation}

\nobreak A ``sequence'' of numbers is just a list of numbers.  For
example, here is a list of numbers:
$$
1,\quad 1,\quad 2,\quad 3,\quad 5,\quad 8,\quad 13,\quad 21,\quad \ldots
$$
Note that numbers in the list can repeat.  And consider those little
dots at the end!  The dots ``\ldots'' signify that the list keeps
going, and going, and going---forever.  Presumably the sequence
continues by following the pattern that the first few ``terms''
suggest.  But what's that pattern?

To make this talk of ``patterns'' less ambiguous, it is useful to
think of a sequence as a function. We have up until now dealt with
functions whose domains are the real numbers, or a subset of the real
numbers, like $f(x)=\sin x$. A sequence is a function with domain the
natural numbers $\N=\{1,2,3,\ldots\}$ or the non-negative integers,
$\ds \Z^{\ge0}=\{0,1,2,3,\ldots\}$. The range of the function is still
allowed to be the real numbers; in symbols, we say that a sequence is
a function $f\colon \N\to\R$.

\marginnote{Maybe you are feeling that this formality is unnecessary,
  or even ridiculous; why can't we just list off a few terms and pick
  up on the pattern intuitively?  As we'll see later, that might be
  very hard---nay, impossible---to do!  There might be very
  different---but equally reasonable---patterns that start the same
  way.

  To resolve this ambiguity, it is perhaps not so ridiculous to
  introduce the formalism of ``functions.''  Functions provide a nice
  language for associating numbers (terms) to other numbers
  (indices).}

Sequences are written in a few different, but equivalent,
ways; you might seen a sequence written as
\begin{align*}
  & a_1,\quad a_2, \quad a_3, \quad\ldots, \\
  & \left(a_n\right)_{n \in \N}, \\
  & \left\{a_n\right\}_{n=1}^\infty, \\
  & \left\{f(n)\right\}_{n=1}^\infty, \quad \mbox{or} \\
  & \left(f(n)\right)_{n \in \N},
\end{align*}

depending on which author you read.  Worse, depending on the
situation, the same author might choose different notation for a
sequence.

Let's summarize the preceding discussion in the following definition.
\begin{definition} \relax\index{sequence} A \defnword{sequence}
  $(a_n)$ is, formally speaking, a real-valued function with domain
  the natural numbers $\N$ or the non-negative integers $\Z^{\ge 0}$.
  Stated more humbly, a sequence assigns a real number to whole
  numbers.

  The ``outputs'' of a sequence are the \defnword{terms} of the
  sequence; the ``$n^{\nth}$ term'' is the real number that the
  sequence associates to the natural number $n$, and is usually
  written $a_n$. \index{sequence!term}.  The $n$ in the phrase
  ``$n^{\nth}$ term'' is called an
  \defnword{index}\index{sequence!index}; the plural of index is
  either indices or indexes, depending on who you ask.
\end{definition} 

\marginnote{Recall that the natural numbers $\N$ are the counting
  numbers $1, 2, 3, 4, \ldots$.  If we want our sequence to start at
  zero, we use $\Z^{\ge 0}$ as the domain instead.  The fancy symbols
  $\Z^{\ge 0}$ refer to the non-negative integers, which include zero
  (since zero is neither positive nor negative) and also positive
  integers (since they certainly aren't negative).}

\marginnote{Some people---perhaps computer scientists?---might include
  zero in the natural numbers $\N$.  Mathematics is cultural.}

\begin{warning}
  Our formal definition of sequence involves $\N$ and $\Z^{\ge 0}$.
  But depending on the context, it may be convenient for a sequence to
  start somewhere else---perhaps with some negative number.  We
  shouldn't let the formal definition get in the way of making the
  best choice for the problem at hand.
\end{warning}

As you can tell, there is a deep tension between precise definition
and a vague flexibility; how we navigate that tension will be a big
part of whether we are successful in this course.  We need to invoke
precision when we're tempted to be too vague, and we need to reach for
an extra helping of vagueness when the formalism is getting in the way
of our understanding.

\section{Defining sequences by giving a rule}

Just as with functions with domain the real numbers, we will most often
encounter sequences that can be expressed by a formula.  We have
already seen the sequence $\ds a_i=f(i)=1-1/2^i$, and others are easy
to come by:
\begin{align*}
  a_i &={i\over i+1} \\
  a_n &={1\over2^n} \\
  a_n &=\sin(n\pi/6) \\
  a_i &={(i-1)(i+2)\over2^i} \\
\end{align*}
Frequently these formulas will make sense if thought of either as
functions with domain $\R$ or $\N$, though occasionally the given
formula will make sense only for whole numbers.

\section{Defining sequences using previous terms}

\marginnote{You might be familiar with \textit{recursion} from, say, a
  computer science course.}

Another way to define a sequence is \textit{recursively}, that is, by
defining the later outputs in terms of previous outputs.  We start by
defining the first few terms of the sequence, and then describe how
later terms are computed in terms of previous terms.

\begin{example}
Define a sequence recursively by
$$
a_1 = 1, \quad a_2 = 3, \quad a_3 = 10,
$$
and the rule that $a_n = a_{n-1} - a_{n-3}$.  Compute $a_5$.
\end{example}

\begin{solution}
  First we compute $a_4$.  Substituting $4$ for $n$ in the rule $a_n = a_{n-1} - a_{n-3}$, we find
$$
a_4 = a_{4-1} - a_{4-3} = a_3 - a_1.
$$
But we have values for $a_3$ and $a_1$, namely $10$ and $1$, respectively.  Therefore $a_4 = 10 - 1 = 9$.

Now we are in a position to compute $a_5$.  Substituting $5$ for $n$ in the rule $a_n = a_{n-1} - a_{n-3}$, we find
$$
a_5 = a_{5-1} - a_{5-3} = a_4 - a_2.
$$
We just computed $a_4 = 9$; we were given $a_2 = 3$.  Therefore $a_5 = 9 - 3 = 6$.
\end{solution}

\marginnote{You can imagine some very complicated sequences defined recursively.  Make up your own sequence and share it with your friends!  Use the \href{https://twitter.com/search?q=\%23sequence}{hashtag \texttt{\#sequence}}.}

\section{Examples}

\marginnote{Tons of entertaining sequences are listed in the \href{http://oeis.org/}{The On-Line Encyclopedia of Integer Sequences}.}

Mathematics proceeds, in part, by finding precise statements for
everyday concepts.  We have already done this for sequences when we
found a precise definition (``function from $\N$ to $\R$'') for the
everyday concept of ``a list of real numbers.''  But all the
formalisms in the world aren't worth the paper they are printed on if
there aren't some interesting \textit{examples} of those precise
concepts.  Indeed, mathematics proceeds not only by generalizing and
formalizing, but also by focusing on specific, concrete instances.
So let me share some specific examples of sequences.

But before I can share these examples, let me address a question: how
can I hand you an example of a sequence? It is not enough just to list
off the first few terms.  Let's see why.

\begin{example}
Consider the sequence $(a_n)$
$$
a_1 = 41, \quad a_2 = 43, \quad a_3 = 47, \quad a_4 = 53, \quad\ldots
$$
What is the next term $a_5$?  Can you identify the sequence?
\end{example}

\marginnote{This particular polynomial $n^4 - n + 41$ is rather interesting, since it outputs many prime numbers.  You can read more at \href{http://oeis.org/A005846}{the OEIS}.}

\begin{solution}
  In spite of many so-called ``intelligence tests'' asking questions
  like this, this question doesn't have an answer.  Or worse, it has
  too many answers!

This sequence might be ``the prime numbers starting at 41.''  In which
case the next term is $a_5 =59$.  But maybe this sequence is given by
the polynomial $a_n = n^2 - n + 41$.  In that case, $a_5 = 61$.  Who
is to say which is the ``better'' answer?
\end{solution}

\marginnote{Recall that a \defnword{prime number} is an integer
  greater than one that has no positive divisors besides itself and
  one.}

Now let's consider two popular ``families'' of sequences.

\subsection{Arithmetic sequences}

The first family\sidenote{Mathematically, the word \defnword{family}
  does not have an entirely precise definition; a family of things is
  a \defnword{collection} or a \defnword{set} of things, but family
  has a connotation of some sort of relatedness.} we consider are the
``arithmetic'' sequences.  Here is a definition.

\begin{definition}
  An \defnword{arithmetic progression} (sometimes called an arithmetic
  sequence)\index{arithmetic progression} is a sequence where each
  term differs from the previous by the same, fixed quantity.
\end{definition}

\begin{example}
  An example of an arithmetic progression is the sequence
  $$
  a_1 = 10, \quad a_2 = 14, \quad a_3 = 18, \quad a_4 = 22, \quad\ldots
  $$
  which is given by the rule $a_n = 6 + 4 \, n$.  Each term differs
  from the previous by four.
\end{example}

In general, an arithmetic progression in which subsequent terms differ
by $m$ can be written as
$$
a_n = m \, (n-1) + a_1.
$$
Alternatively, we could describe an arithmetic progression
recursively, by giving a starting value $a_1$, and using the rule that
$a_{n} = a_{n-1} + m$.

\marginnote{Why are arithmetic progressions called \textit{arithmetic?}  Note that every term is the \defnword{arithmetic mean}, that is, the \defnword{average}, of its two neighbors.}

An arithmetic progression can decrease; for instance,
$$
17,\quad  15,\quad  13,\quad  11,\quad  9, \quad\ldots
$$
is an arithmetic progression.

\subsection{Geometric sequences}

The second family we consider are geometric progressions.

\begin{definition}
  A \defnword{geometric progression} (sometimes called an geometric
  sequence)\index{geometric progression} is a sequence where the ratio
  between subsequent terms if the same, fixed quantity.
\end{definition}

\begin{example}
  An example of an geometric progression is the sequence
  $$
  a_1 = 10, \quad a_2 = 30, \quad a_3 = 90, \quad a_4 = 270, \quad\ldots
  $$
  which is given by the rule $a_n = 10 \cdot 3^n$.  Each term is three
  times the preceding term.
\end{example}

In general, a geometric progression in which the ratio between
subsequent terms is $r$ can be written as
$$
a_n = a_1 \cdot r^{n-1}.
$$
Alternatively, we could describe a geometric progression
recursively, by giving a starting value $a_1$, and using the rule that
$a_{n} = r \cdot a_{n-1}$.

\marginnote{Why are geometric progressions called \textit{geometric?}  Note that every term is the \defnword{geometric mean} of its two neighbors.  The geometric mean of two numbers $a$ and $b$ is defined to be $\sqrt{ab}$.

Of course, that raises another question: why is the geometric mean called \textit{geometric?}  One geometric interpretation of the geometric mean of $a$ and $b$ is this: the geometric mean is the side length of a square whose area is equal to that of the rectangle having side lengths $a$ and $b$.}

A geometric progression needn't be increasing.  For instance, in the following geometric progression
$$
\frac{7}{5}, \quad \frac{7}{10}, \quad \frac{7}{20}, \quad \frac{7}{40}, \quad \frac{7}{80}, \quad \frac{7}{160}, \quad\ldots
$$
the ratio between subsequent terms is one half, and each term is smaller than the previous.

%\subsection{Triangular numbers}
%BADBAD

\subsection{Fibonacci numbers}

\marginnote{The Fibonacci numbers are interesting enough that a
  journal, \href{http://www.fq.math.ca/}{The Fibonacci Quaterly} is
  published four times yearly entirely on topics related to the
  Fibonacci numbers.}

The \defnword{Fibonacci numbers} are defined recursively, starting with
$$
F_0 = 0 \mbox{ and } F_1 = 1
$$
and the rule that $F_{n} = F_{n-1} + F_{n-2}$.  We can restate this
formula in words, instead of symbols; stated in words, each term is
the sum of the previous two terms.  So the sequence of Fibonacci
numbers begins 
$$
0, \quad 1, \quad 1, \quad 2, \quad 3, \quad 5, \quad 8, \quad 13, \quad 21, \quad 35, \quad\ldots
$$
and continues.

This is certainly not the last time we will see the Fibonacci numbers.

\subsection{Collatz sequence}

Here is fun sequence with which to amuse your friends.  Let's start
our sequence with $a_1 = 6$.  Subsequent terms are defined using the rule
$$
a_n = \begin{cases} a_{n-1} / 2 & \mbox{ if $a_{n-1}$ is even, and } \\
3 \, a_{n-1} + 1 & \mbox{ if $a_{n-1}$ is odd.}
\end{cases}
$$
Let's compute $a_2$.  Since $a_1$ is even, we follow the instructions
in the first line, to find that $a_2 = a_1/2 = 3$. To compute $a_3$,
note that $a_2$ is odd so we follow the instruction in the second
line, and $a_3 = 3 \, a_2 + 1 = 3 \cdot 3 + 1 = 10$.  Since $a_3$ is
even, the first line applies, and $a_4 = a_3 / 2 = 10 / 2 = 5$.  But
$a_4$ is odd, so the second line applies, and we find $a_5 = 3 \cdot 5
+ 1 = 16$.  And $a_5$ is even, so $a_6 = 16 / 2 = 8$.  And $a_6$ is
even, so $a_7 = 8/4 = 4$.  And $a_7$ is even, so $a_8 = 4 / 2 = 2$,
and then $a_9 = 2/2 = 1$.  Oh, but $a_9$ is odd, so $a_{10} = 3 \cdot
1 + 1 = 4$.  And it repeats.  Let's write down the start of this sequence:
$$
6,\quad %1 
3,\quad %2
10,\quad  %3
5,\quad  %4
16,\quad  %5
8,\quad  %6
4,\quad  %7
2,\quad  %8
1,\quad  %9
4,\quad %10
2,\quad %11
1,\quad %12
\overbrace{4,\quad %10
2,\quad %11
1,}^{\mbox{repeats}}\quad %12
4,\quad %10
\ldots
$$
What if we had started with a number other than six?  What if we set
$a_1 = 25$ but then we used the same rule?  In that case, since $a_1$
is odd, we compute $a_2$ by finding $3 \, a_1 + 1 = 3 \cdot 25 + 1 =
76$.  Since $76$ is even, the next term is half that, meaning $a_3 =
38$.  If we keep this up, we find that our sequence begins
\begin{align*}
&25,\quad 76,\quad 38,\quad 19,\quad 58,\quad 29,\quad 88,\quad 44,\quad 22,\quad 11,\quad 34,\quad 17,\quad 52,\quad 26, \\
&13,\quad 40,\quad 20,\quad 10,\quad 5,\quad 16,\quad 8,\quad 4,\quad 2, \quad 1, \quad \ldots
\end{align*}
and then it repeats ``4, 2, 1, 4, 2, 1, \ldots'' just like before.

\marginnote{If you think you have an argument that answers the Collatz conjecture, I challenge you to try your hand at the $5x+1$ conjecture, that is, use the rule
$
a_n = \displaystyle\begin{cases} a_{n-1} / 2 & \mbox{ if $a_{n-1}$ is even, and } \\
5 \, a_{n-1} + 1 & \mbox{ if $a_{n-1}$ is odd.}
\end{cases}
$}

Does this always happen?  Is it true that no matter which positive
integer you start with, if you apply the half-if-even, $3x+1$-if-odd
rule, you end up getting stuck in the ``4, 2, 1, \ldots'' loop?  That
this is true is the \defnword{Collatz conjecture}; it has been
verified for all starting values below $5 \times 2^{60}$.  Nobody has
found a value which doesn't return to one, but for all anybody knows
there \textit{might} be a larger starting value which doesn't return
to one---nobody knows either way.  It is an unsolved
problem\sidenote{This is not the last unsolved problems we will
  encounter in this course.  There are many things which humans do not
  understand.} in mathematics.

\section{Where is a sequence headed?  Take a limit!}

We've seen a lot of sequences, and already there are a few things we
might notice.  For instance, the arithmetic progression
$$
1,\quad 8,\quad 15,\quad 22,\quad 29,\quad 36,\quad 43,\quad 50,\quad 57,\quad 64,\quad 71,\quad 78,\quad 85,\quad 92,\quad \ldots
$$
just keeps getting bigger and bigger.  No matter how large a number
you think of, if I add enough $7$'s to $1$, eventually I will surpass
the giant number you thought of.  On the other hand, the terms in a geometric progression where each term is half the previous term, namely
$$
\frac{1}{2},\quad \frac{1}{4},\quad \frac{1}{8},\quad \frac{1}{16},\quad \frac{1}{32},\quad \frac{1}{64},\quad \frac{1}{128},\quad \frac{1}{256},\quad \frac{1}{512},\quad \frac{1}{1024},\quad \ldots ,
$$
are getting closer and closer to zero.  No matter how close you stand
near but not at zero, eventually this geometric sequence gets even closer than you
are to zero.

These two sequences have very different stories.  One shoots off to
infinity; the other zooms in towards zero.  Mathematics is not just
about numbers; mathematics provides tools for talking about the
qualitative features of the numbers we deal with.  What about two
sequences we just considered?  They are qualitatively very different.  The first ``goes to''
infinity; the second ``goes to'' zero.

\marginnote{If you were with us in Calculus One, you are perhaps
  already guessing that by ``goes to,'' I actually mean ``has
  limit.''}

In short, given a sequence, it is helpful to be able to say something
qualitative about it; we may want to address the question such as
``what happens after a while?'' Formally, when faced with a sequence,
we are interested in the limit
$$\lim_{i\to \infty} f(i) = \lim_{i\to\infty} a_i.$$
In Calculus One, we studied a similar question about
$$\lim_{x\to\infty} f(x)$$
when $x$ is a variable taking on real values; now, in Calculus Two, we
simply want to restrict the ``input'' values to be integers. No
significant difference is required in the definition of limit, except
that we specify, perhaps implicitly, that the variable is an integer.

\begin{definition} \relax\index{limit of a sequence}
Suppose that $\ds\left(a_n\right)_{n \in \N}$ is a sequence.
To say that $\ds \lim_{n\to \infty}a_n=L$ is to say that \\
\null\quad for every $\epsilon>0$, \\
\null\quad\quad there is an $N > 0$, \\
\null\quad so that whenever $n>N$, \\
\null\quad\quad we have $|a_n-L|<\epsilon$. \\
If $\ds \lim_{n\to\infty}a_n=L$ we say that the sequence
\defnword{converges}\index{convergent
  sequence}\index{sequence!convergent}.  If there is no value $L$ so
that $\ds \lim_{n\to\infty}a_n = L$, then we say that the limit
\defnword{does not exist}, or equivalently that the sequence
\defnword{diverges}\index{divergent sequence}\index{sequence!divergent}.
\end{definition} 

\marginnote{The definition of limit is being written as if it were
  poetry, what with line breaks and all.  Like the best of poems, it deserves to be memorized, performed, internalized.  Humanity struggled for millenia to find the wisdom contained therein.}

One way to compute the limit of a sequence is to compute the limit of
a function.
\begin{theorem}
  Let $f(x)$ be a real-valued function.  If $a_n = f(n)$ defines a
  sequence and $\ds \lim_{x\to\infty}f(x)=L$ in the sense of Calculus
  One, then $\ds \lim_{n\to\infty} a_n=L$ as well.
\end{theorem}

\begin{example}
Since $\ds \lim_{x\to\infty}(1/x)=0$, it is
clear that also $\ds \lim_{n\to\infty}(1/n)=0$; in other words, the sequence of numbers
$${1\over1},\quad {1\over2},\quad {1\over3},\quad {1\over4},\quad {1\over5},\quad {1\over6},\quad \ldots$$
get closer and closer to 0, or more precisely, as close as you want to get to zero, after a while, all the terms in the sequence are that close.
\end{example}

But it is important to note that the converse\sidenote{The
  \defnword{converse} of a statement is what you get when you swap the
  assumption and the conclusion; the converse of ``if it is raining,
  then it is cloudy'' is the statement ``if it is cloudy, then it is
  raining.''  Which of those statements are true?} of this theorem is
not true.  To show the converse is not true, it is enough to provide a
single example where it fails.  Here's the counterexample\sidenote{An
  instance of (a potential) general rule being broken is called a
  \defnword{counterexample}.  This is a popular term among mathematicians and philosophers.}.

\begin{example}
  Consider the sequence $a_n = f(n)=\sin(n\pi)$.  This is the sequence
$$
  \sin(0\pi),\quad \sin(1\pi),\quad\sin(2\pi),\quad\sin(3\pi),\quad\ldots,
$$
which is just the sequence $0, 0, 0, 0, \ldots$ since $\sin(n\pi)=0$
whenever $n$ is an integer.  Since the sequence is just the constant sequence, we have
$$
\lim_{n\to\infty} f(n)= \lim_{n\to\infty} 0 = 0. 
$$. But $\ds \lim_{x\to\infty}f(x)$, when $x$ is real, does not exist: as $x$ gets
bigger and bigger, the values $\sin(x\pi)$ do not get closer and
closer to a single value, but instead oscillate between $-1$ and $1$.
\end{example} 

Here's some general advice. If you want to know $\ds \lim_{n\to\infty}
a_n$, you might first think of a function $f(x)$ where $a_n = f(n)$,
and then attempt to compute $\ds \lim_{x\to\infty}f(x)$.  If the limit
of the function exists, then it is equal to the limit of the sequence.
But, if for some reason $\ds \lim_{x\to\infty}f(x)$ does not exist, it
may nevertheless still be the case that $\ds \lim_{n\to\infty}f(n)$
exists---you'll just have to figure out another way to compute it.


\begin{marginfigure}[0in]
\begin{tikzpicture}
	\begin{axis}[
            domain=0:20,
            ymax=2.25,
            ymin=-1.5,
            xmin=0,
            xmax=20.25,
            axis lines =middle, xlabel={$x$ and $n$}, ylabel=$y$,
            every axis y label/.style={at=(current axis.above origin),anchor=south},
            every axis x label/.style={at=(current axis.right of origin),anchor=west}
          ]
%scale = 10 ; print(join([str((n(x/scale,digits=5),n(cos(pi*(x/scale)) + (4/5)^(x/scale),digits=5))) for x in range(1,20*scale)],' '))
          \addplot [penColor, smooth] plot coordinates { (0.50000, 0.89443) (0.60000, 0.56567) (0.70000, 0.26760) (0.80000, 0.027494) (0.90000, -0.13300) (1.0000, -0.20000) (1.1000, -0.16871) (1.2000, -0.043935) (1.3000, 0.16041) (1.4000, 0.42267) (1.5000, 0.71554) (1.6000, 1.0088) (1.7000, 1.2721) (1.8000, 1.4782) (1.9000, 1.6055) (2.0000, 1.6400) (2.1000, 1.5769) (2.2000, 1.4211) (2.3000, 1.1863) (2.4000, 0.89437) (2.5000, 0.57243) (2.6000, 0.25078) (2.7000, -0.040337) (2.8000, -0.27365) (2.9000, -0.42750) (3.0000, -0.48800) (3.1000, -0.45035) (3.2000, -0.31936) (3.3000, -0.10893) (3.4000, 0.15926) (3.5000, 0.45795) (3.6000, 0.75686) (3.7000, 1.0258) (3.8000, 1.2373) (3.9000, 1.3699) (4.0000, 1.4096) (4.1000, 1.3516) (4.2000, 1.2007) (4.3000, 0.97085) (4.4000, 0.68364) (4.5000, 0.36636) (4.6000, 0.049256) (4.7000, -0.23742) (4.8000, -0.46638) (4.9000, -0.61598) (5.0000, -0.67232) (5.1000, -0.63061) (5.2000, -0.49564) (5.3000, -0.28131) (5.4000, -0.0093174) (5.5000, 0.29309) (5.6000, 0.59564) (5.7000, 0.86808) (5.8000, 1.0831) (5.9000, 1.2191) (6.0000, 1.2621) (6.1000, 1.2074) (6.2000, 1.0597) (6.3000, 0.83294) (6.4000, 0.54878) (6.5000, 0.23447) (6.6000, -0.079722) (6.7000, -0.36355) (6.8000, -0.58973) (6.9000, -0.73661) (7.0000, -0.79029) (7.1000, -0.74597) (7.2000, -0.60845) (7.3000, -0.39163) (7.4000, -0.11721) (7.5000, 0.18757) (7.6000, 0.49245) (7.7000, 0.76718) (7.8000, 0.98445) (7.9000, 1.1226) (8.0000, 1.1678) (8.1000, 1.1151) (8.2000, 0.96947) (8.3000, 0.74469) (8.4000, 0.46246) (8.5000, 0.15006) (8.6000, -0.16227) (8.7000, -0.44427) (8.8000, -0.66867) (8.9000, -0.81381) (9.0000, -0.86578) (9.1000, -0.81979) (9.2000, -0.68066) (9.3000, -0.46224) (9.4000, -0.18626) (9.5000, 0.12005) (9.6000, 0.42642) (9.7000, 0.70259) (9.8000, 0.92129) (9.9000, 1.0608) (10.000, 1.1074) (10.100, 1.0560) (10.200, 0.91171) (10.300, 0.68821) (10.400, 0.40722) (10.500, 0.096038) (10.600, -0.21510) (10.700, -0.49595) (10.800, -0.71920) (10.900, -0.86321) (11.000, -0.91410) (11.100, -0.86704) (11.200, -0.72687) (11.300, -0.50745) (11.400, -0.23045) (11.500, 0.076831) (11.600, 0.38415) (11.700, 0.66128) (11.800, 0.88087) (11.900, 1.0213) (12.000, 1.0687) (12.100, 1.0182) (12.200, 0.87474) (12.300, 0.65205) (12.400, 0.37187) (12.500, 0.061465) (12.600, -0.24891) (12.700, -0.52903) (12.800, -0.75153) (12.900, -0.89484) (13.000, -0.94502) (13.100, -0.89728) (13.200, -0.75644) (13.300, -0.53635) (13.400, -0.25874) (13.500, 0.049172) (13.600, 0.35710) (13.700, 0.63484) (13.800, 0.85501) (13.900, 0.99603) (14.000, 1.0440) (14.100, 0.99405) (14.200, 0.85108) (14.300, 0.62889) (14.400, 0.34924) (14.500, 0.039337) (14.600, -0.27055) (14.700, -0.55015) (14.800, -0.77223) (14.900, -0.91508) (15.000, -0.96482) (15.100, -0.91663) (15.200, -0.77537) (15.300, -0.55485) (15.400, -0.27684) (15.500, 0.031470) (15.600, 0.33979) (15.700, 0.61788) (15.800, 0.83845) (15.900, 0.97983) (16.000, 1.0281) (16.100, 0.97856) (16.200, 0.83594) (16.300, 0.61412) (16.400, 0.33476) (16.500, 0.025176) (16.600, -0.28440) (16.700, -0.56376) (16.800, -0.78547) (16.900, -0.92802) (17.000, -0.97748) (17.100, -0.92903) (17.200, -0.78748) (17.300, -0.56673) (17.400, -0.28842) (17.500, 0.020141) (17.600, 0.32871) (17.700, 0.60711) (17.800, 0.82785) (17.900, 0.96947) (18.000, 1.0180) (18.100, 0.96867) (18.200, 0.82625) (18.300, 0.60463) (18.400, 0.32549) (18.500, 0.016113) (18.600, -0.29326) (18.700, -0.57244) (18.800, -0.79395) (18.900, -0.93632) (19.000, -0.98559) (19.100, -0.93695) (19.200, -0.79523) (19.300, -0.57430) (19.400, -0.29584) (19.500, 0.012890) (19.600, 0.32162) (19.700, 0.60019) (19.800, 0.82107) (19.900, 0.96285)};
          \node at (axis cs:2.2000, 1.4211) [anchor=south west] {\color{penColor}$f(x)$};  
          \node at (axis cs:10, 1.15) [anchor=south] {\color{penColor2}$a_n$};
% print(join([str((x,n(cos(pi*x) + (4/5)^x,digits=5))) for x in range(1,20)],' ')) 
          \addplot[color=penColor2,fill=penColor2,only marks,mark=*] coordinates{(1, -0.20000) (2, 1.6400) (3, -0.48800) (4, 1.4096) (5, -0.67232) (6, 1.2621) (7, -0.79029) (8, 1.1678) (9, -0.86578) (10, 1.1074) (11, -0.91410) (12, 1.0687) (13, -0.94502) (14, 1.0440) (15, -0.96482) (16, 1.0281) (17, -0.97748) (18, 1.0180) (19, -0.98559) (20, 1.0115)};

        \end{axis}
\end{tikzpicture}
\caption{Plots of $f(x) = \cos (\pi \, x) + (4/5)^x$ and the sequence $a_n = (-1)^n + (4/5)^n$.}
\label{fig:graphs-of-sequences}
\end{marginfigure}

\section{Graphs}

It is occasionally useful to think of the graph of a sequence. Since
the function is defined only for integer values, the graph is just a
sequence of dots. In Figure~\xrefn{fig:graphs-of-sequences} we see the
graph of a sequences and the graph of a corresponding real-valued
function.

There are lots of real-valued functions which ``fill in'' the missing
values of a sequence.

\begin{example}
  Here's a particularly tricky example of ``filling in'' the missing values of a sequence.  Consider the sequence
  $$
  1,\quad 2,\quad 6,\quad 24,\quad 120,\quad 720,\quad 5040,\quad 40320,\quad 362880,\quad\ldots,
  $$
  where the $n^{\nth}$ term is the product of the first $n$ integers.
  In other words $a_n = n!$, where the exclamation mark denotes the
  \defnword{factorial} function.  Explicitly describe a function $f$
  of a real variable $x$, so that $a_n = f(n)$ for natural numbers
  $n$.
\end{example}

\begin{marginfigure}[0in]
\begin{tikzpicture}
	\begin{axis}[
            domain=0:4.2,
            ymax=25,
            ymin=-0.1,
            xmin=0,
            xmax=4.2,
            axis lines =middle, xlabel={$x$ and $n$}, ylabel=$y$,
            every axis y label/.style={at=(current axis.above origin),anchor=south},
            every axis x label/.style={at=(current axis.right of origin),anchor=west}
          ]
          \node at (axis cs:2.5, 3) [anchor=south] {\color{penColor}$f(x)$};  
          \addplot [very thick, penColor, domain=(0:1)] {1};
          \addplot [very thick, penColor, domain=(1:2)] {1};
          \addplot [very thick, penColor, domain=(2:3)] {2};
          \addplot [very thick, penColor, domain=(3:4)] {6};
          \addplot [very thick, penColor, domain=(4:5)] {24};
          \addplot[color=penColor,fill=penColor,only marks,mark=*] coordinates{(1, 1) (2, 2) (3, 6) (4, 24)};
          \addplot[color=penColor,fill=background,only marks,mark=*] coordinates{(2, 1) (3, 2) (4, 6) (5, 24)};
% print(join([str((x,factorial(x))) for x in range(1,20)],' ')) 
          \addplot[color=penColor2,fill=penColor2,only marks,mark=*] coordinates{(1, 1) (2, 2) (3, 6) (4, 24)};
          \node at (axis cs:3, 7) [anchor=south] {\color{penColor2}$a_n$};
        \end{axis}
\end{tikzpicture}
\caption{A plot of $f(x) = \floor{x}!$ and $a_n = n!$.  Recall that, by convention, $0! = 1$.}
\label{fig:floor-graph}
\end{marginfigure}

\marginnote{It is hard to define a ``greatest integer'' function,
  because they are all pretty great.}

\begin{solution}
  There are lots of solutions.  Here is a solution:
$$
f(x) = \floor{x}!.
$$
In that definition, $\floor{x}$ denotes the ``greatest integer less
  than or equal to $x$'' and is called the \defnword{floor function}.  This is shown in Figure~\xrefn{fig:floor-graph}.

  On the other hand, there are much trickier things that you could try
  to do.  If you define the \defnword{Gamma function}
  $$
  \Gamma(z) = \int_0^\infty t^{z-1} e^{-t} \, dt.
  $$
  then it is perhaps very surprising to find out that $g(x) =
  \Gamma(x+1)$ is a function so that $g(n) = n!$ for natural numbers
  $n$.  A graph is shown in Figure~\xrefn{fig:gamma-function}.
  Unlike $f$, which fails to be continuous, the function $g$ is continuous.
\end{solution}


\begin{marginfigure}[0in]
\begin{tikzpicture}
	\begin{axis}[
            domain=0:4.2,
            ymax=25,
            ymin=-0.1,
            xmin=0,
            xmax=4.2,
            axis lines =middle, xlabel={$x$ and $n$}, ylabel=$y$,
            every axis y label/.style={at=(current axis.above origin),anchor=south},
            every axis x label/.style={at=(current axis.right of origin),anchor=west}
          ]
%scale = 30 ; print(join([str((n(x/scale,digi>s=5),n(gamma(x/scale + 1),digits=5))) for x in range(0,4.2*scale)],' '))
\addplot [penColor, smooth] plot coordinates { 
(0.00000, 1.0000) (0.033333, 0.98183) (0.066667, 0.96566) (0.10000, 0.95135) (0.13333, 0.93874) (0.16667, 0.92772) (0.20000, 0.91817) (0.23333, 0.91000) (0.26667, 0.90312) (0.30000, 0.89747) (0.33333, 0.89298) (0.36667, 0.88959) (0.40000, 0.88726) (0.43333, 0.88595) (0.46667, 0.88561) (0.50000, 0.88623) (0.53333, 0.88776) (0.56667, 0.89020) (0.60000, 0.89352) (0.63333, 0.89770) (0.66667, 0.90275) (0.70000, 0.90864) (0.73333, 0.91538) (0.76667, 0.92296) (0.80000, 0.93138) (0.83333, 0.94066) (0.86667, 0.95078) (0.90000, 0.96177) (0.93333, 0.97362) (0.96667, 0.98636) (1.0000, 1.0000) (1.0333, 1.0146) (1.0667, 1.0300) (1.1000, 1.0465) (1.1333, 1.0639) (1.1667, 1.0823) (1.2000, 1.1018) (1.2333, 1.1223) (1.2667, 1.1440) (1.3000, 1.1667) (1.3333, 1.1906) (1.3667, 1.2158) (1.4000, 1.2422) (1.4333, 1.2699) (1.4667, 1.2989) (1.5000, 1.3293) (1.5333, 1.3612) (1.5667, 1.3946) (1.6000, 1.4296) (1.6333, 1.4662) (1.6667, 1.5046) (1.7000, 1.5447) (1.7333, 1.5867) (1.7667, 1.6306) (1.8000, 1.6765) (1.8333, 1.7245) (1.8667, 1.7748) (1.9000, 1.8274) (1.9333, 1.8823) (1.9667, 1.9398) (2.0000, 2.0000) (2.0333, 2.0629) (2.0667, 2.1287) (2.1000, 2.1976) (2.1333, 2.2697) (2.1667, 2.3451) (2.2000, 2.4240) (2.2333, 2.5065) (2.2667, 2.5930) (2.3000, 2.6834) (2.3333, 2.7782) (2.3667, 2.8773) (2.4000, 2.9812) (2.4333, 3.0900) (2.4667, 3.2040) (2.5000, 3.3233) (2.5333, 3.4485) (2.5667, 3.5796) (2.6000, 3.7170) (2.6333, 3.8611) (2.6667, 4.0122) (2.7000, 4.1707) (2.7333, 4.3369) (2.7667, 4.5112) (2.8000, 4.6942) (2.8333, 4.8862) (2.8667, 5.0877) (2.9000, 5.2993) (2.9333, 5.5215) (2.9667, 5.7549) (3.0000, 6.0000) (3.0333, 6.2575) (3.0667, 6.5282) (3.1000, 6.8126) (3.1333, 7.1116) (3.1667, 7.4260) (3.2000, 7.7567) (3.2333, 8.1044) (3.2667, 8.4704) (3.3000, 8.8554) (3.3333, 9.2606) (3.3667, 9.6871) (3.4000, 10.136) (3.4333, 10.609) (3.4667, 11.107) (3.5000, 11.632) (3.5333, 12.185) (3.5667, 12.767) (3.6000, 13.381) (3.6333, 14.029) (3.6667, 14.711) (3.7000, 15.431) (3.7333, 16.191) (3.7667, 16.992) (3.8000, 17.838) (3.8333, 18.730) (3.8667, 19.673) (3.9000, 20.667) (3.9333, 21.718) (3.9667, 22.828) (4.0000, 24.000) (4.0333, 25.239) (4.0667, 26.548) (4.1000, 27.932) (4.1333, 29.395) (4.1667, 30.942)};
          \node at (axis cs:2.5, 3.3233) [anchor=south east] {\color{penColor}$f(x)$};  
% print(join([str((x,factorial(x))) for x in range(1,20)],' ')) 
          \addplot[color=penColor2,fill=penColor2,only marks,mark=*] coordinates{(1, 1) (2, 2) (3, 6) (4, 24)};
          \node at (axis cs:3, 7) [anchor=north west] {\color{penColor2}$a_n$};
        \end{axis}
\end{tikzpicture}
\caption{Plots of $f(x) = \int_0^\infty t^{z} e^{-t} \, dt.$ and $a_n = n!$.}
\label{fig:gamma-function}
\end{marginfigure}


\section{Helpful theorems about limits}

Not surprisingly, the properties of limits of real functions translate
into properties of sequences quite easily. 

%\vbox{
\begin{theorem} \relax\label{thm:properties of sequences}
Suppose that $\ds\lim_{n\to\infty}a_n=L$ and 
$\ds\lim_{n\to\infty}b_n=M$ and
$k$ is some constant. Then
\begin{align*}
&\lim_{n\to\infty} ka_n = k\lim_{n\to\infty}a_n=kL, \\
&\lim_{n\to\infty} (a_n+b_n) = \lim_{n\to\infty}a_n+\lim_{n\to\infty}b_n=L+M, \\
&\lim_{n\to\infty} (a_n-b_n) = \lim_{n\to\infty}a_n-\lim_{n\to\infty}b_n=L-M, \\
&\lim_{n\to\infty} (a_nb_n) = \lim_{n\to\infty}a_n\cdot\lim_{n\to\infty}b_n=LM, \mbox{ and}\\
&\lim_{n\to\infty} {a_n\over b_n} = {\lim_{n\to\infty}a_n\over
  \lim_{n\to\infty}b_n}={L\over M},\hbox{ provided $M \neq 0$.} \\
\end{align*}
\end{theorem}
%}

Likewise, there is an analogue of the squeeze theorem for functions.

\begin{theorem}\relax\label{thm:squeeze theorem for sequences}
Suppose there is some $N$ so that for all $n > N$, it is the case that $\ds a_n \le b_n \le c_n$. If $$\ds\lim_{n\to\infty}a_n=\ds\lim_{n\to\infty}c_n=L$$, 
then $\ds\lim_{n\to\infty}b_n=L$.
\end{theorem}

And a final useful fact:

\begin{theorem} \relax\label{thm:absolute value sequence}
$\ds\lim_{n\to\infty}|a_n|=0$ if and only if
$\ds\lim_{n\to\infty}a_n=0$.
\end{theorem}
\marginnote{Sometimes people write ``iff'' as shorthand for ``if and only if.''}

This says simply that the size of $\ds a_n$ gets close to zero if and
only if $\ds a_n$ gets close to zero.

\begin{example}
Determine whether the sequence $a_n = \frac{n}{n+1}$ converges or
diverges. If it converges, compute the limit.
\end{example}

\begin{solution}
Consider the real-valued function
$$
f(x) = \frac{x}{x+1}.
$$
Since $a_n = f(n)$, it will be enough to find $\ds\lim_{x \to \infty} f(x)$ in order to find $\ds\lim_{n \to \infty} a_n$.
We compute, as in Calculus One, that
\begin{align*}
\lim_{x\to\infty}{x\over x+1}
&= \lim_{x\to\infty} \frac{(x+1) - 1}{x+1} \\
&= \lim_{x\to\infty} \left( \frac{x+1}{x+1} - \frac{1}{x+1} \right) \\
&= \lim_{x\to\infty} \left( 1 - \frac{1}{x+1} \right) \\
&= \lim_{x\to\infty} 1 - \lim_{x \to \infty} \frac{1}{x+1} \\
&= 1 - \lim_{x \to \infty} \frac{1}{x+1} = 1 - 0 = 1. \\
\end{align*}
We therefore conclude that $\ds\lim_{n \to \infty} a_n = 1$.
\end{solution}
\marginnote{And this is reasonable: by choosing $n$ to be a large enough whole number, I can make $\ds\frac{n}{n+1}$ as close to $1$ as I would like.  Just imagine how close $\frac{10000000000}{10000000001}$ is to one.}

\begin{example}%BADBAD
Determine whether $\ds\bigg\{{\ln n\over n}\bigg\}_{n=1}^\infty$ converges or
diverges. If it converges, compute the limit.
\end{example}

\begin{solution}%BADBAD
We compute
$$\lim_{x\to\infty}{\ln x\over x}=\lim_{x\to\infty}{1/x\over 1}=
0,$$
using L'H\^opital's Rule. 
Thus the sequence converges to 0.
\end{solution}%BADBAD

\begin{example}%BADBAD
Determine whether $\ds\{(-1)^n\}_{n=0}^\infty$ converges or
diverges. If it converges, compute the limit. This does not make sense
for all real exponents, but the sequence is easy to understand: it is
$$1,-1,1,-1,1\ldots$$
and clearly diverges.
\end{example}
\label{example:alternating ones}

\begin{example}%BADBAD
Determine whether $\ds\{(-1/2)^n\}_{n=0}^\infty$ converges or
diverges. If it converges, compute the limit.
\end{example}
\begin{solution}We consider the sequence 
$\ds\{|(-1/2)^n|\}_{n=0}^\infty=\{(1/2)^n\}_{n=0}^\infty$.
Then
$$
  \lim_{x\to\infty}\left({1\over2}\right)^x=\lim_{x\to\infty}{1\over2^x}=0,
$$
so by theorem~\xrefn{thm:absolute value sequence} the sequence converges to 0.
\end{solution}

\begin{example}%BADBAD
Determine whether $\ds\{(\sin n)/\sqrt{n}\}_{n=1}^\infty$ converges or
diverges. If it converges, compute the limit. 
\end{example}

\begin{solution}%BADBAD
Since $|\sin n|\le 1$, $\ds 0\le|\sin n/\sqrt{n}|\le
1/\sqrt{n}$ and we can use theorem~\xrefn{thm:squeeze theorem for
sequences} with $\ds a_n=0$ and $\ds c_n=1/\sqrt{n}$. Since
$\ds\lim_{n\to\infty} a_n=\ds\lim_{n\to\infty} c_n=0$, 
$\ds\lim_{n\to\infty}\sin n/\sqrt{n}=0$ and the sequence converges to 0.
\end{solution}

\begin{example}%BADBAD
A particularly common and useful sequence is $\ds \{r^n\}_{n=0}^\infty$,
for various values of $r$.  For which values of $r$ does this sequence converge?
\end{example}

\begin{solution}%BADBAD
Some are quite easy to understand: If $r=1$
the sequence converges to 1 since every term is 1, and likewise if
$r=0$ the sequence converges to 0. If $r=-1$ this is
the sequence of example~\xrefn{example:alternating ones} and
diverges. If $r>1$ or $r<-1$ the terms $\ds r^n$ get large without limit,
so the sequence diverges. If $0<r<1$ then the sequence converges to
0. If $-1<r<0$ then $\ds |r^n|=|r|^n$ and $0<|r|<1$, so the sequence
$\ds \{|r|^n\}_{n=0}^\infty$ converges to 0, so also 
$\ds\{r^n\}_{n=0}^\infty$ converges to 0.
converges. In summary, $\ds \{r^n\}$ converges precisely when
$-1<r\le1$ in which case
$$
  \lim_{n\to\infty} r^n=\begin{cases}
  0& if $-1<r<1$ \\
  1& if $r=1$ \end{cases}
$$
\vskip-10pt\end{solution}

\section{Qualitative features of limits}

\marginnote{Your first exposure to mathematics might have been about
  \defnword{constructions}; you might have been asked to find an
  answer or to propose a solution to a problem.  But much of
  mathematics is concerned with showing \defnword{existence}, even if
  the thing that is being shown to exist cannot be exhibited itself.}

Sometimes we will not be able to determine the limit of a sequence,
but we still would like to know whether it converges.  In many cases,
we can determine whether a limit exists, without needing to---or
without even being able to---compute that limit.

\subsection{Monotonicity}

And sometimes we don't even care about limits, but we'd simply like
some terminology with which to describe features we might notice about
sequences.  Here is some of that terminology.

\marginnote{For instance, how much money I have on day $n$ is a
  sequence; I probably hope that sequence is an increasing sequence.}

\begin{definition}
  A sequence is called
  \defnword{increasing}\index{sequence!increasing} (or sometimes
  \defnword{strictly increasing}) if $\ds a_n<a_{n+1}$ for all $n$.
  It is called {\dfont
    non-decreasing\index{sequence!non-decreasing}\/} if $\ds a_n\le
  a_{n+1}$ for all $n$.

  Similarly a sequence is {\dfont
    decreasing\index{sequence!decreasing}\/} (or, by some people,
  \defnword{strictly decreasing}) if $\ds a_n>a_{n+1}$ for all $n$ and
  {\dfont non-increasing\index{sequence!non-increasing}\/} if $\ds
  a_n\ge a_{n+1}$ for all $n$.
\end{definition}
To make matters worse, the people who insist on saying ``strictly
increasing'' may---much to everybody's confusion---insist on calling a
non-decreasing sequence ``increasing.'' I'm not going to play their
game; I'll be careful to say ``non-decreasing'' when I mean a sequence
which is getting larger or staying the same.

To make matters better, lots of facts are true for sequences which are
either increasing or decreasing; to talk about this situation without
constantly saying ``either increasing or decreasing,'' we can make up
a single word to cover both cases.
\begin{definition}
  If a sequence is increasing, non-decreasing, decreasing, or
  non-increasing, it is said to be {\dfont
    monotonic\index{sequence!monotonic}\/}.
\end{definition}

Let's see some examples of sequences which are monotonic.
\begin{example}
The sequence $\ds a_n = {2^n-1\over2^n}$ which starts
$$
  {1\over2},\quad {3\over4},\quad {7\over8},\quad {15\over16},\quad \ldots,
$$
is increasing.  On the other hand, the sequence $\ds b_n = {n+1\over n}$, which starts
$$ 
  {2\over1},\quad{3\over2},\quad{4\over3},\quad{5\over4},\quad\ldots,
$$
is decreasing.
\end{example}

\subsection{Boundedness}%BADBAD

A sequence is {\dfont bounded above\index{sequence!bounded above}\/}
if there is some number $N$ such that $\ds a_n\le N$ for every $n$,
and {\dfont bounded below\index{sequence!bounded below}\/} if there is
some number $N$ such that $\ds a_n\ge N$ for every $n$. If a sequence
is bounded above and bounded below it is {\dfont
bounded\index{sequence!bounded}\/}. If a sequence $\ds
\{a_n\}_{n=0}^\infty$ is increasing or non-decreasing it is bounded
below (by $\ds a_0$), and if it is decreasing or non-increasing it is
bounded above (by $\ds a_0$).  Finally, with all this new terminology
we can state an important theorem.

\begin{theorem} If a sequence is bounded and monotonic then it converges.
\end{theorem}

We will not prove this; the proof appears in many calculus books. It
is not hard to believe: suppose that a sequence is increasing and
bounded, so each term is larger than the one before, yet never larger
than some fixed value $N$. The terms must then get closer and closer
to some value between $\ds a_0$ and $N$. It need not be $N$, since $N$ may
be a ``too-generous'' upper bound; the limit will be the
smallest number that is above all of the terms $\ds a_i$.

\begin{example}
All of the terms $\ds (2^i-1)/2^i$ are less than 2, and the sequence is
increasing. As we have seen, the limit of the sequence is 1---1 is the
smallest number that is bigger than all the terms in the sequence.
Similarly, all of the terms $(n+1)/n$ are bigger than $1/2$, and the
limit is 1---1 is the largest number that is smaller than the terms of
the sequence.
\end{example}

We don't actually need to know that a sequence is monotonic to apply
this theorem---it is enough to know that the sequence is
``eventually'' monotonic, that is, that at some point it becomes
increasing or decreasing. For example, the sequence $10$, $9$, $8$,
$15$, $3$, $21$, $4$, $3/4$, $7/8$, $15/16$, $31/32,\ldots$ is not
increasing, because among the first few terms it is not. But starting
with the term $3/4$ it is increasing, so the theorem tells us that the
sequence $3/4, 7/8, 15/16, 31/32,\ldots$ converges.  Since convergence
depends only on what happens as $n$ gets large, adding a few
terms at the beginning can't turn a convergent sequence into a
divergent one.

\begin{example}
Show that $\ds\{n^{1/n}\}$ converges. 
\par\nobreak\ssk\noindent
We first show that 
this sequence is decreasing, that is, that $\ds n^{1/n}>
(n+1)^{1/(n+1)}$. Consider the real function $\ds f(x)=x^{1/x}$ when
$x\ge1$. We can compute the derivative, $\ds f'(x)=x^{1/x}(1-\ln x)/x^2$,
and note that when $x\ge 3$ this is negative. Since the function has
negative slope, $\ds n^{1/n}>
(n+1)^{1/(n+1)}$ when $n\ge 3$. Since all terms of the sequence are
positive, the sequence is decreasing and bounded when $n\ge3$, and so
the sequence converges. (As it happens, we can compute the limit in
this case, but we know it converges even without knowing the limit; see
exercise~\xrefn{exercise:exponential limit}.)
\end{example}

\begin{example}
Show that $\ds\{n!/n^n\}$ converges.
\par\nobreak\ssk\noindent
Again we show that the sequence is decreasing, and since each term is
positive the sequence converges. We can't take the derivative this
time, as $x!$ doesn't make sense for $x$ real. But we note that if 
$\ds a_{n+1}/a_n < 1$ then $\ds a_{n+1}< a_n$, which is what we want to
know. So we look at $\ds a_{n+1}/a_n$:
$$ 
  {a_{n+1}\over a_n} = {(n+1)!\over (n+1)^{n+1}}{n^n\over n!}=
  {(n+1)!\over n!}{n^n\over (n+1)^{n+1}}=
  {n+1\over n+1}\left({n\over n+1}\right)^n=
  \left({n\over n+1}\right)^n < 1.
$$
(Again it is possible to compute the limit; see
exercise~\xrefn{exercise:factorial limit}.)
\end{example}

\begin{exercises}

\begin{exercise} \label{exercise:exponential limit}
Compute $\ds\lim_{x\to\infty} x^{1/x}$.
\begin{answer} $1$
\end{answer}\end{exercise}

\begin{exercise} Use the squeeze theorem to show that 
$\ds\lim_{n\to\infty} {n!\over n^n}=0$.
\label{exercise:factorial limit}
\end{exercise}

\begin{exercise} Determine whether $\ds\{\sqrt{n+47}-\sqrt{n}\}_{n=0}^\infty$ 
converges or diverges. If it converges, compute the limit.
\begin{answer} $0$
\end{answer}\end{exercise}

\begin{exercise} Determine whether 
$\ds\left\{{n^2+1\over (n+1)^2}\right\}_{n=0}^\infty$ 
converges or diverges. If it converges, compute the limit.
\begin{answer} $1$
\end{answer}\end{exercise}

\begin{exercise} Determine whether 
$\ds\left\{{n+47\over\sqrt{n^2+3n}}\right\}_{n=1}^\infty$ 
converges or diverges. If it converges, compute the limit.
\begin{answer} $1$
\end{answer}\end{exercise}

\begin{exercise} Determine whether 
$\ds\left\{{2^n\over n!}\right\}_{n=0}^\infty$ 
converges or diverges. 
\begin{answer} $0$
\end{answer}\end{exercise}

\end{exercises}

