
\nobreak

If all of the terms $\ds a_n$ in a series are non-negative, then clearly 
the sequence of partial sums $\ds s_n$ is non-decreasing. This means that
if we can show that the sequence of partial sums is bounded, the
series must converge. We know that if the series converges, the terms
$\ds a_n$ approach zero, but this does not mean that $\ds a_n\ge a_{n+1}$ for
every $n$. Many useful and interesting series do have this property,
however, and they are among the easiest to understand. Let's look at
an example.

\subsection{An example}

\begin{example} Show that $\ds\sum_{n=1}^\infty {1\over n^2}$ converges.
\end{example}

\begin{marginfigure}[0in]
\begin{tikzpicture}
	\begin{axis}[
            domain=0.5:5.5,
            ymax=1.05,
            ymin=-0.05,
            xmin=0,
            xmax=5.5,
            axis lines =middle, xlabel={$x$ and $n$}, ylabel=$y$,
            every axis y label/.style={at=(current axis.above origin),anchor=south},
            every axis x label/.style={at=(current axis.right of origin),anchor=west}
          ]
          \addplot [penColor, smooth, very thick,domain=(0.75:5.5)] {1/(x*x)};
          \node [anchor=west,penColor] at (axis cs:1.5,0.5) {$f(x) = 1/x^2$};

          \draw[penColor2] (axis cs:0,0) rectangle (axis cs:1,1);
          \node [anchor=north,penColor2] at (axis cs:0.5,1) {$a_1$};
          \draw[penColor2] (axis cs:1,0) rectangle (axis cs:2,0.25);
          \node [anchor=north,penColor2] at (axis cs:1.5,0.25) {$a_2$};
          \draw[penColor2] (axis cs:2,0) rectangle (axis cs:3,0.11111111);
          \node [anchor=north,penColor2] at (axis cs:2.5,0.11111111) {$a_3$};
          \draw[penColor2] (axis cs:3,0) rectangle (axis cs:4,0.0625);
          \node [anchor=south,penColor2,yshift=2pt] at (axis cs:3.5,0.0625) {$a_4$};
          \draw[penColor2] (axis cs:4,0) rectangle (axis cs:5,0.04);
          \node [anchor=south,penColor2,yshift=2pt] at (axis cs:4.5,0.04) {$a_5$};
          \draw[penColor2] (axis cs:5,0) rectangle (axis cs:6,0.027777777);

        \end{axis}
\end{tikzpicture}
\caption{Plot of $f(x) = 1/x^2$ alongside boxes representing $a_n = 1/n^2$.}
\label{fig:integral-test-for-one-over-n-squared}
\end{marginfigure}

\begin{solution}
  We have actually already seen this very series---it appeared in
  Example~\xrefn{example:basel-problem}, where we attacked this series
  via condensation.  Here, we consider a different approach, namely
  integration.

  The terms $\ds 1/n^2$ are positive and decreasing, and since
  $\ds\lim_{x\to\infty} 1/x^2=0$, the terms $\ds 1/n^2$ approach
  zero. We seek an upper bound for all the partial sums, that is, we
  want to find a number $N$ so that $s_n\le N$ for every $n$. The
  upper bound is provided courtesy of integration.

  Figure~\xrefn{fig:integral-test-for-one-over-n-squared} shows the
  graph of $\ds y=1/x^2$ together with some rectangles that lie
  completely below the curve and that all have base length
  one. Because the heights of the rectangles are determined by the
  height of the curve, the areas of the rectangles are $\ds 1/1^2$,
  $\ds 1/2^2$, $\ds 1/3^2$, and so on---in other words, exactly the
  terms of the series. The partial sum $\ds s_n$ is simply the sum of
  the areas of the first $n$ rectangles. Because the rectangles all
  lie between the curve and the $x$-axis, any sum of rectangle areas
  is less than the corresponding area under the curve, and so of
  course any sum of rectangle areas is less than the area under the
  entire curve, that is, all the way to infinity.  There is a bit of
  trouble at the left end, where there is an asymptote, but we can
  work around that easily. Here it is:
$$
  s_n={1\over 1^2}+{1\over 2^2}+{1\over 3^2}+\cdots+{1\over n^2}
  < 1 + \int_1^n {1\over x^2}\,dx < 1+\int_1^\infty {1\over x^2}\,dx 
  =1+1=2,
$$
recalling how to compute this improper integral. Since the sequence of partial
sums $\ds s_n$ is increasing and bounded above by 2, we know that 
$\ds\lim_{n\to\infty}s_n=L<2$, and so the series converges to some
number less than 2.
\end{solution}

In fact, it is possible, though difficult, to show that $\ds
\sum_{n=1}^\infty \frac{1}{n^2} = \pi^2/6$.

\subsection{Harmonic series}
\label{subsection:harmonic-series-by-integrating}

We already know that $\sum 1/n$ diverges. What goes wrong if we try to
apply this technique to it? We find
$$
  s_n={1\over 1}+{1\over 2}+{1\over 3}+\cdots+{1\over n}
  < 1 + \int_1^n {1\over x}\,dx < 1+\int_1^\infty {1\over x}\,dx 
$$
but this amounts to saying nothing, because the improper integral
$\int_1^\infty {1\over x}\,dx$ does not converge, and claiming that
$s_n$ is bounded by something divergent is to make no claim at all:
every real number is less than infinity!  In other words, this does
{\em not\/} prove that $\sum 1/n$ diverges; it is just that this
particular calculation fails to prove that it converges. A slight
modification, however, allows us to prove in a second way that $\sum
1/n$ diverges.

\begin{example} Consider a slightly altered version of Figure~\xrefn{fig:integral-test-for-one-over-n-squared}, shown in Figure~\xrefn{fig:integral-test-for-one-over-n}.  Explain how to use the figure to see that the harmonic series diverges.
\end{example}

\begin{marginfigure}[0in]
\begin{tikzpicture}
	\begin{axis}[
            domain=0.5:6,
            ymax=1.25,
            ymin=-0.05,
            xmin=0.75,
            xmax=6.25,
            axis lines =middle, xlabel={$x$ and $n$}, ylabel=$y$,
            every axis y label/.style={at=(current axis.above origin),anchor=south},
            every axis x label/.style={at=(current axis.right of origin),anchor=west}
          ]
          \addplot [penColor, smooth, very thick,domain=(0.75:6.25)] {1/(x)};
          \node [anchor=west,penColor] at (axis cs:3,1.1) {$f(x) = 1/x$};

          \draw[penColor2] (axis cs:1,0) rectangle (axis cs:2,1);
          \node [anchor=south,penColor2] at (axis cs:1.5,1) {$a_1$};
          \draw[penColor2] (axis cs:2,0) rectangle (axis cs:3,0.5);
          \node [anchor=south,penColor2] at (axis cs:2.5,0.5) {$a_2$};
          \draw[penColor2] (axis cs:3,0) rectangle (axis cs:4,0.333333333);
          \node [anchor=south,penColor2] at (axis cs:3.5,0.333333333) {$a_3$};
          \draw[penColor2] (axis cs:4,0) rectangle (axis cs:5,0.25);
          \node [anchor=south,penColor2] at (axis cs:4.5,0.25) {$a_4$};
          \draw[penColor2] (axis cs:5,0) rectangle (axis cs:6,0.2);
          \node [anchor=south,penColor2] at (axis cs:5.5,0.2) {$a_5$};
          \draw[penColor2] (axis cs:6,0) rectangle (axis cs:7,0.16666666);

        \end{axis}
\end{tikzpicture}
\caption{Plot of $f(x) = 1/x$ alongside boxes representing $a_n = 1/n$.}
\label{fig:integral-test-for-one-over-n}
\end{marginfigure}


\begin{solution}
The rectangles this time are above the curve, that is, each rectangle
completely contains the corresponding area under the curve. This means
that 
$$s_n = {1\over 1}+{1\over 2}+{1\over 3}+\cdots+{1\over n}
> \int_1^{n+1} {1\over x}\,dx = \ln x\Big|_1^{n+1}=\ln(n+1).$$
As $n$ gets bigger, $\ln(n+1)$ goes to infinity, so the sequence of
partial sums $\ds s_n$ must also go to infinity, so the harmonic series
diverges. 
\end{solution}

\subsection{Statement of integral test}

The important fact that clinches this example is that
$$\lim_{n\to\infty} \int_1^{n+1} {1\over x}\,dx = \infty,$$
which we can rewrite as
$$\int_1^\infty {1\over x}\,dx = \infty.$$
So these two examples taken together indicate that we can prove that a
series converges or prove that it diverges with a single calculation
of an improper integral. This is known as the {\dfont integral
  test\index{integral test}\index{series!integral test}\/}, 
which we state as a theorem.

\begin{theorem}[Integral test]\label{thm:integral-test}
 Suppose that $f(x)>0$ and is decreasing on the infinite interval
$[k,\infty)$ (for some $k\ge1$)
and that $\ds a_n=f(n)$. Then the series
$\ds\sum_{n=k}^\infty a_n$ converges if and only if the improper
integral $\ds\int_{k}^\infty f(x)\,dx$ converges.
\end{theorem}

\subsection{$p$-series}
\label{subsection:p-series-via-integration}

The two examples we have seen are examples of
$p$-series\index{p@$p$-series}\index{series!$p$-series}, which we first encountered in Subsection~\xrefn{subsection:p-series}.
Recall that a $p$-series is
any series of the form $\ds \sum_{n=1}^\infty 1/n^p$.

\begin{theorem}\label{thm:p-series} A $p$-series converges if and only if $p>1$.
\end{theorem}

We already proved this theorem using condensation in Example~\xrefn{example:p-series-p-leq-1} and Example~\xrefn{example:p-series-p-ge-1}.  Nevertheless, we provide a second proof, using the integral test.

\begin{proof}
We use the integral test; the case $p=1$ is that of the harmonic series, which we know diverges, so without loss of generality, we may assume that
$p \neq 1$.   If $p\le0$, $\ds\lim_{n\to\infty} 1/n^p\not=0$, so the series diverges by Theorem~\xrefn{thm:nth-term-test}. So we may also assume that $p > 0$.

Then we compute
$$
  \int_1^{\infty} {1\over x^p}\,dx=\lim_{N\to\infty} \left.{x^{1-p}\over
  1-p}\right|_{1}^N=\lim_{N\to\infty} {N^{1-p}\over 1-p}-{1\over 1-p}.
$$
If $p>1$ then $1-p<0$ and $\ds\lim_{N\to\infty}N^{1-p}=0$, so the
  integral converges. If $0<p<1$ then $1-p>0$ and 
$\ds\lim_{N\to\infty}N^{1-p}=\infty$, so the integral diverges.
\end{proof}

\begin{example} Show that $\ds\sum_{n=1}^\infty {1\over {n^3}}$ converges. 
\end{example}

\begin{solution}
We could of course use
the integral test, but now that we have the theorem we may simply note
that this is a $p$-series with $p = 3>1$.
\end{solution}

\begin{example} Show that $\ds\sum_{n=1}^\infty {5\over n^4}$ converges. 
\end{example}
\begin{solution}
We know that if
$\ds \sum_{n=1}^\infty 1/n^4$ converges then $\ds \sum_{n=1}^\infty 5/n^4$
also converges, by Theorem~\xrefn{thm:series-are-linear-one}. Since 
$\ds \sum_{n=1}^\infty 1/n^4$ is a convergent $p$-series, 
 $\ds \sum_{n=1}^\infty 5/n^4$ converges also.
\end{solution}

\begin{example} Show that $\ds\sum_{n=1}^\infty {5\over \sqrt{n}}$ diverges.
\end{example}
\begin{solution}
This also follows from
Theorem~\xrefn{thm:series-are-linear}.  Since $\ds\sum_{n=1}^\infty
{1\over \sqrt{n}}$ is a $p$-series with $p=1/2<1$, it diverges, and so
does $\ds\sum_{n=1}^\infty {5\over \sqrt{n}}$.  
\end{solution}

\subsection{Integrating for approximations}

Since it is typically difficult to compute the value of a series
exactly, a good approximation is frequently required. In a real sense,
a good approximation is only as good as we know it is, that is, while
an approximation may in fact be good, it is only valuable in practice
if we can guarantee its accuracy to some degree\sidenote{After all,
  $\pi \approx 17$ just with very bad error bounds.  It is better to
  make a statement like $|\pi - 17| < 14$, which is not only saying
  that $\pi$ is ``close'' to 17, but is quantifying exactly how close
  (within 14---so perhaps not all that close).}. This guarantee is
usually easy to come by for series with decreasing positive terms.

\begin{example} Approximate $\ds \sum_{n=1}^\infty 1/n^2$ to two decimal places.
\end{example}

\marginnote{It turns out that $\sum_{n=1}^\infty \frac{1}{n^2} =
  \frac{\pi^2}{6}$, so your approximation of this series will
  also---in a roundabout way---yield an approximate value for
  $\pi$---and one which will be better than 17.}

\begin{solution}
Referring to Figure~\xrefn{fig:integral-test-for-one-over-n-squared},
if we approximate the sum by $\ds \sum_{n=1}^N 1/n^2$, the error we make is the
total area of the remaining rectangles, all of which lie under the
curve $\ds 1/x^2$ from $x=N$ out to infinity. So we know the true value of
the series is larger than the approximation, and no bigger than the
approximation plus the area under the curve from $N$ to
infinity. Roughly, then, we need to find $N$ so that 
$$\int_N^\infty {1\over x^2}\,dx < 1/100.$$
We can compute the integral:
$$\int_N^\infty {1\over x^2}\,dx = {1\over N},$$ 
so $N=100$ is a good starting point.  Adding up the first 100 terms
gives approximately $1.634983900$, and that plus $1/100$ is
$1.644983900$, so approximating the series by the value halfway
between these will be at most $1/200=0.005$ in error.  The midpoint is
$1.639983900$, but while this is correct to $\pm0.005$, we can't tell
if the correct two-decimal approximation is $1.63$ or $1.64$. We need
to make $N$ big enough to reduce the guaranteed error, perhaps to
around $0.004$ to be safe, so we would need $1/N\approx 0.008$, or
$N=125$. Now the sum of the first 125 terms is approximately
$1.636965982$, and that plus $0.008$ is $1.644965982$ and the point
halfway between them is $1.640965982$. The true value is then
$1.640965982\pm 0.004$, and all numbers in this range round to $1.64$,
so $1.64$ is correct to two decimal places. 
\end{solution}

Since $\sum_{n=1}^\infty \frac{1}{n^2} = \frac{\pi^2}{6}$, our estimate yields
$$
1.63 < \frac{\pi^2}{6} < 1.65,
$$
and so
$$
9.78 < \pi^2 < 9.90.
$$
which means that
$$
3.127 < \pi < 3.147,
$$
which is better than $\pi \approx 3$.  And we have explicit bounds on the error.

\begin{exercises}

Determine whether each series converges or diverges.

\twocol

\begin{exercise} $\ds\sum_{n=1}^\infty {1\over n^{\pi/4}}$
\begin{answer} diverges
\end{answer}\end{exercise}

\begin{exercise} $\ds\sum_{n=1}^\infty {n\over n^2+1}$
\begin{answer} diverges
\end{answer}\end{exercise}

\begin{exercise} $\ds\sum_{n=1}^\infty {\ln n\over n^2}$
\begin{answer} converges
\end{answer}\end{exercise}

\begin{exercise} $\ds\sum_{n=1}^\infty {1\over n^2+1}$
\begin{answer} converges
\end{answer}\end{exercise}

\begin{exercise} $\ds\sum_{n=1}^\infty {1\over e^n}$
\begin{answer} converges
\end{answer}\end{exercise}

\begin{exercise} $\ds\sum_{n=1}^\infty {n\over e^n}$
\begin{answer} converges
\end{answer}\end{exercise}

\begin{exercise} $\ds\sum_{n=2}^\infty {1\over n\ln n}$
\begin{answer} diverges
\end{answer}\end{exercise}

\begin{exercise} $\ds\sum_{n=2}^\infty {1\over n(\ln n)^2}$
\begin{answer} converges
\end{answer}\end{exercise}

\endtwocol

\msk
\begin{exercise} Find an $N$ so that
$\ds\sum_{n=1}^\infty {1\over n^4}$ is between
$\ds\sum_{n=1}^N {1\over n^4}$ and
$\ds\sum_{n=1}^N {1\over n^4} + 0.005$.
\begin{answer} $N=5$
\end{answer}\end{exercise}

\begin{exercise} Find an $N$ so that
$\ds\sum_{n=0}^\infty {1\over e^n}$ is between
$\ds\sum_{n=0}^N {1\over e^n}$ and
$\ds\sum_{n=0}^N {1\over e^n} + 10^{-4}$.
\begin{answer} $N=10$
\end{answer}\end{exercise}

\begin{exercise} Find an $N$ so that
$\ds\sum_{n=1}^\infty {\ln n\over n^2}$ is between
$\ds\sum_{n=1}^N {\ln n\over n^2}$ and
$\ds\sum_{n=1}^N {\ln n\over n^2} + 0.005$.
\begin{answer} $N=1687$
\end{answer}\end{exercise}

\begin{exercise} Find an $N$ so that
$\ds\sum_{n=2}^\infty {1\over n(\ln n)^2}$ is between
$\ds\sum_{n=2}^N {1\over n(\ln n)^2}$ and
$\ds\sum_{n=2}^N {1\over n(\ln n)^2} + 0.005$.
\begin{answer} any integer greater than $\ds e^{200}$
\end{answer}\end{exercise}

\end{exercises}

