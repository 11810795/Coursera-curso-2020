\PassOptionsToPackage{table}{xcolor}
\documentclass[justified]{tufte-book}
\usepackage{mooculus}

\usepackage{mathtools}
\DeclarePairedDelimiter{\ceil}{\lceil}{\rceil}
\DeclarePairedDelimiter{\floor}{\lfloor}{\rfloor}

%\usepackage{showkeys} %% Useful for debugging

\setcounter{secnumdepth}{2}
\usepackage{ulem}
\usepackage{amssymb}
\newcommand{\N}{\mathbb{N}}
\newcommand{\Z}{\mathbb{Z}}


% Prints the month name (e.g., January) and the year (e.g., 2008)
\newcommand{\monthyear}{%
  \ifcase\month\or January\or February\or March\or April\or May\or June\or
  July\or August\or September\or October\or November\or
  December\fi\space\number\year
}

% Generates the index
\usepackage{makeidx}
\makeindex

\newcommand{\xrefn}[1]{\ref{#1}}

\newenvironment{lemma}{\subsection*{Lemma}}{}
\newenvironment{remark}[1]{\subsection*{Remark: #1}}{}

%\def\exam{\null}
\def\pagerdef{\null}

\def\dfont{\bf}
\def\em{\it}           % for emphasis

\newcommand{\ds}{\displaystyle}

\let\ssk\smallskip \let\msk\medskip \let\bsk\bigskip

\usepackage{multicol}
\def\twocol{\begin{multicols}{2}}
\def\endtwocol{\end{multicols}}

\title{Sequences and Series}
%\author{Jim Fowler and Bart Snapp}
\publisher{This document was typeset on \today.}
%\newcommand{\blankpage}{\newpage\hbox{}\thispagestyle{empty}\newpage}

%% % Prints an epigraph and speaker in sans serif, all-caps type.
%% \newcommand{\openepigraph}[2]{%
%%   %\sffamily\fontsize{14}{16}\selectfont
%%   \begin{fullwidth}
%%   \sffamily\large
%%   \begin{doublespace}
%%   \noindent\allcaps{#1}\\% epigraph
%%   \noindent\allcaps{#2}% author
%%   \end{doublespace}
%%   \end{fullwidth}
%% }


\begin{document}
\frontmatter

\maketitle

\pagebreak

% v.4 copyright page

\begin{fullwidth}
~\vfill
\thispagestyle{empty}
\setlength{\parindent}{0pt}
\setlength{\parskip}{\baselineskip}
Copyright \copyright\ \the\year\ Jim Fowler and Bart Snapp

This work is licensed under the Creative Commons
Attribution-NonCommercial-ShareAlike License. To view a copy of this
license, visit
\url{http://creativecommons.org/licenses/by-nc-sa/3.0/}~or send a
letter to Creative Commons, 543 Howard Street, 5th Floor, San
Francisco, California, 94105, USA. If you distribute this work or a
derivative, include the history of the document. 

The source code is available
at: \url{https://github.com/kisonecat/sequences-and-series/tree/master/textbook}

\noindent
This text is based on David Guichard's open-source calculus text which
in turn is a modification and expansion of notes written by Neal
Koblitz at the University of Washington. David Guichard's text is
available at \url{http://www.whitman.edu/mathematics/calculus/}~under a Creative Commons license.

\noindent The book includes some exercises and examples from {\it
  Elementary Calculus: An Approach Using Infinitesimals}, by H.~Jerome
Keisler, available at
\url{http://www.math.wisc.edu/~keisler/calc.html}~under a Creative
Commons license. In addition, the chapter on differential equations
and the section on numerical integration are largely derived from the
corresponding portions of Keisler's book.  Albert Schueller, Barry
Balof, and Mike Wills have contributed additional material.  Thanks to
Walter Nugent, Nicholas J.~Roux, Dan Dimmitt, Donald Wayne Fincher, Nathalie
Dalp\'e, chas, Clark Archer, Sarah Smith, MithrandirAgain, AlmeCap,
and hrzhu for proofreading and
\href{https://github.com/kisonecat/sequences-and-series/graphs/contributors}{contributing
  on GitHub}.

\noindent This book is typeset in the Kerkis font, 
Kerkis \copyright~Department of Mathematics, University of the Aegean.


\noindent We will be glad to receive corrections and suggestions for
improvement at \texttt{fowler@math.osu.edu} or
\texttt{snapp@math.osu.edu}.

\end{fullwidth}

\tableofcontents


%% \renewcommand{\listtheoremname}{List of Main Theorems}
%% \setcounter{tocdepth}{1}
%% \listoftheorems[numwidth=4em,ignoreall,show={mainTheorem}]



%\chapter*{List of Main Theorems}% uses ntheorem
%\theoremlisttype{opt} 
%\listtheorems{mainTheorem}


%\listoffigures

%\listoftables

% r.7 dedication
%\cleardoublepage
%~\vfill
%\begin{doublespace}
%\thispagestyle{empty}
%\noindent\fontsize{18}{22}\selectfont\itshape
%\nohyphenation
%\centerline{\it For Kathleen,\/}
%\centerline{\it without whose encouragement\/}
%\centerline{\it this book would not have\/}
%\centerline{\it been written.\/}
%\end{doublespace}
%\vfill
%\vfill

% r.9 introduction
%\cleardoublepage

\chapter*{How to \sout{read} do mathematics}

Reading mathematics is \textbf{not} the same as reading a novel---it's
more fun, and more interactive!  To read mathematics you need
\begin{enumerate}
\item a pen,
\item plenty of blank paper, and
\item the courage to write down everything---even ``obvious'' things.
\end{enumerate}
As you read a math book, you work along with me, the author, trying to
anticipate my next thoughts, repeating many of the same calculations I
did to write this book.  You must \textbf{write} down each expression,
\textbf{sketch} each graph, and constantly \textbf{think} about what
you are doing.  You should work examples.  You should fill-in the
details I left out.  This is not an easy task; it is \textbf{hard}
work, but, work that is, I very much hope, rewarding in the end.

Mathematics is not a passive endeavor.  I may call you a ``reader''
but you are not reading; you are writing this book for yourself.

\vspace{0.5in}
\hfill---the so-called ``author''

\chapter*{Acknowledgments}

This text is a modification of
\href{http://www.whitman.edu/mathematics/calculus/}{David Guichard's
  open-source calculus text} which was itself a modification of notes
written by Neal Koblitz at the University of Washington and includes
exercises and examples from {\it Elementary Calculus: An Approach
  Using Infinitesimals} by H.~Jerome Keisler.  I am grateful to David
Guichard for choosing a
\href{http://creativecommons.org/licenses/by-nc-sa/3.0/}{Creative
  Commons} license.  Albert Schueller, Barry Balof, and Mike Wills
have contributed additional material.  The stylesheet, based on \texttt{tufte-latex}, was designed by Bart Snapp.

This textbook was specifically used for a
\href{https://www.coursera.org/course/sequence}{Coursera course}
called ``Calculus Two: Sequences and Series.''  Many thanks go to
Walter Nugent, Donald Wayne Fincher, Robert Pohl, chas,
\href{https://github.com/clark-archer}{Clark Archer}, Mikhail, Sarah Smith,
Mavaddat Javid, Grigoriy Mikhalkin, Susan Stewart, Donald Eugene Parker, Francisco Alonso Sarr\'ia, Eduard Pascual Saez, Lam Tin-Long, mrBB, Hanna
Szabelska, Arthur Dent, Ryan Noble, and
\href{https://github.com/hrzhu}{hrzhu} for finding and correcting
errors in early editions of this text.  Thank you!

\vspace{0.5in}
\hfill---Jim Fowler

% Start the main matter (normal chapters)
\mainmatter

%%%%%%%%%%%%%%%%%%%%%%%%%%%%%%%%%%%%%%%%%%%%%%%%%%%%%%%%%%%%%%%%
\chapter*{Introduction, or\ldots what is this all about?}

\input introduction

%%%%%%%%%%%%%%%%%%%%%%%%%%%%%%%%%%%%%%%%%%%%%%%%%%%%%%%%%%%%%%%%
\chapter{Sequences}
\label{chapter:sequences}

%\begin{outline}
%\noindent\startcontents[section]% kind
%\printcontents[section]{}{1}{}% kind, prefix, top, init-code
%\end{outline}

\input sequences

%%%%%%%%%%%%%%%%%%%%%%%%%%%%%%%%%%%%%%%%%%%%%%%%%%%%%%%%%%%%%%%%
\chapter{Series}
\label{chapter:series}

%\begin{outline}
%\noindent\startcontents[section]% kind
%\printcontents[section]{}{1}{}% kind, prefix, top, init-code
%\end{outline}
 
\input series

%%%%%%%%%%%%%%%%%%%%%%%%%%%%%%%%%%%%%%%%%%%%%%%%%%%%%%%%%%%%%%%%
\chapter{Convergence tests}
\label{chapter:convergence-tests}

%\begin{outline}
%\startcontents[convergence-tests]% kind
%\printcontents[convergence-tests]{}{1}{}% kind, prefix, top, init-code
%\end{outline}

It is generally quite difficult---indeed, often impossible---to
determine the value of a series exactly.  Even if we can't compute the
value of a series, in many cases it is possible to determine whether
or not the series converges. We will spend most of our time on this
problem.


\section{Ratio tests}
\label{section:ratio-test}
\input ratio-test

\section{Integral test}
\label{section:integral-test}
\input integral-test

\section{More comparisons}
\label{section:more-comparison-tests}
\input more-comparison-tests

\section{The mostly useless root test}
\label{section:root-test}
\input root-test

%%%%%%%%%%%%%%%%%%%%%%%%%%%%%%%%%%%%%%%%%%%%%%%%%%%%%%%%%%%%%%%%
\chapter{Alternating series}
\label{chapter:alternating-series}

%\startcontents[sections]% kind
%\printcontents[sections]{p}{1}{}% kind, prefix, top, init-code
%\end{contents}

\section{Absolute convergence}
\label{section:absolute-convergence}
\input absolute-convergence

\section{Alternating series test}
\label{section:alternating-series-test}
\input alternating-series

%%%%%%%%%%%%%%%%%%%%%%%%%%%%%%%%%%%%%%%%%%%%%%%%%%%%%%%%%%%%%%%%
\chapter{Another comparison test}

We've covered a ton of material thus far in this course; there is one
more comparison test that comes in quite handy---the Limit Comparison
Test---which we will meet in Section~\xrefn{section:limit-comparison-test}.
The purpose of this chapter, however, runs deeper than ``just''
another comparison test.  

The emphasis on series has been almost entirely on the question of
their convergence; we have not paid much heed to the value of the
series, but we've developed a lot of techniques to analyze their
convergence.  The question is always ``Does it converge?'' and the
answer is ``yes, it converges!'' or ``no, it does not converge.''
Considering how qualitative our answer is, we might hope that there
are equally qualitative methods for analyzing series.  If convergence
is just a yes-or-no matter, one might hope that the methods for
analyzing series are equally loose and qualitative.

But that hasn't been our experience.  Convergence is a tricky
business, requiring precision and careful analysis.  There have been
hints, though, that things are easier than they seem: the comparison
test is perhaps the best example of that.  In your past mathematical
life, you've probably been given ``expressions'' or ``equations'' to
which you apply various rules in order to derive an answer.  With the
comparison test, the situation is less about rules, and more about
creatively ignoring parts of the expression in order to find a useful
bound.  More than being rule-based, testing convergence via the
comparison test requires some guesswork, and a willingness to ignore
the parts of the expression that don't matter, in order to get at the
part that does.

Let me be more precise.  Suppose we wanted to analyze the convergence of a series such as
\[
\sum_{n=52}^\infty \frac{n^4 - 3n + 5}{2n^5 + 5n^3 - n^2}
\]
This is a complicated series, but the given expression is the ratio
between a fifth degree polynomial and a fourth degree polynomial, so this series ``is more or less'' the same as the series
\[
\sum_{n=52}^\infty \frac{n^4}{n^5} = \sum_{n=52}^\infty \frac{1}{n}
\]
which is the harmonic series, and diverges!  This is how we'd like to
think, but we need to justify the concept of ``is more or less.''
This is what Section~\xrefn{section:limit-comparison-test} will teach
us to do.

\section{Convergence depends on the tail}
\label{section:convergence-for-tails}
\input convergence-for-tails

\section{Limit comparison test}
\label{section:limit-comparison-test}
\input limit-comparison-test

%%%%%%%%%%%%%%%%%%%%%%%%%%%%%%%%%%%%%%%%%%%%%%%%%%%%%%%%%%%%%%%%
\chapter{Power series}
\label{chapter:power-series}

\input power-series

\section{Calculus with power series}
\label{section:calculus-with-power-series}
\input term-by-term

%%%%%%%%%%%%%%%%%%%%%%%%%%%%%%%%%%%%%%%%%%%%%%%%%%%%%%%%%%%%%%%%
\chapter{Taylor series}
\label{chapter:taylor-series}

We have seen that some functions can be represented as
series.  But thus far, our only examples have been those that result
from manipulation of our one fundamental example, the geometric
series.  We might start with
$$
\frac{1}{1-x} = \sum_{n=0}^\infty x^n \mbox{ when $|x| < 1$,}
$$
and then, say, integrate term-by-term to get a formula for a
logarithm, as we did in
Example~\xrefn{example:formula-for-log-three-halves}.  Instead of
starting with a series representing a function, and then messing
around with the series to find more functions represented by series,
we should \textit{start} with a function, and then try to find a
series that represents it---if that is possible!

\section{Finding Taylor series}
\label{section:finding-taylor-series}
\input taylor-series

\section{Taylor's Theorem}
\label{section:taylors-theorem}
\input remainders

%%%%%%%%%%%%%%%%%%%%%%%%%%%%%%%%%%%%%%%%%%%%%%%%%%%%%%%%%%%%%%%
\chapter*{Review}
\input review

%%%%%%%%%%%%%%%%%%%%%%%%%%%%%%%%%%%%%%%%%%%%%%%%%%%%%%%%%%%%%%%%
 \chapter*{Epilogue (or\ldots what happens to Harry?)}

 The worst part about reading a great\sidenote{Or even a bad novel.}
 novel is that last page.  The book gets thinner and thinner, and
 then, poof!  Not just the characters, but the whole world that the
 author has crafted for them is gone!  And how frequently I want to
 stay in that world just a bit longer.

 Humanity has found an antidote to novel-endings; this antidote is the
 \textit{sequel.}  With a name like ``Calculus Two'' you might be
 getting the idea that there is a Calculus Three, and who knows\ldots
 maybe!

 A-hundred-and-some-odd\sidenote{Or even.}-pages ago, I pointed out
 that reading mathematics is \textbf{not} the same as reading a novel.
 This book is done, but you are not.  There is more mathematics yet to
 learn about, and more mathematics yet to create.  And I don't mean to
 say that you should write go and write fan fiction.  I'm no author,
 and you are no mere reader.  You have worked through the exercises,
 you have thought about this material in your own way---so you are the
 author of your own understanding, and you must keep writing.

\vspace{0.5in}
\hfill---the so-called ``author''

%\bibliography{sample-handout}
%\bibliographystyle{plainnat}

\finalizeanswers
\chapter*{Answers to Exercises}
\small
\addcontentsline{toc}{chapter}{Answers to Exercises}
\IfFileExists{answers.tex}{\subsection *{Answers for 1.8}
\hypertarget {a:1.8.1}{\hyperlink {e:1.8.1}{\bfseries 1.}} \mdseries $1$\qquad 
\hypertarget {a:1.8.3}{\hyperlink {e:1.8.3}{\bfseries 3.}} \mdseries $0$\qquad 
\hypertarget {a:1.8.4}{\hyperlink {e:1.8.4}{\bfseries 4.}} \mdseries $1$\qquad 
\hypertarget {a:1.8.5}{\hyperlink {e:1.8.5}{\bfseries 5.}} \mdseries $1$\qquad 
\hypertarget {a:1.8.6}{\hyperlink {e:1.8.6}{\bfseries 6.}} \mdseries $0$\qquad 
}
\normalsize
\backmatter

%\addcontentsline{toc}{chapter}{Index}
\printindex


\end{document}


