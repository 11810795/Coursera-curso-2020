
\nobreak A bounded, monotonic sequence necessarily converges
(Theorem~\xrefn{thm:bounded-monotonic}).  How does this fact about
sequences relate to series?  When is the sequence of partial sums
monotonic?  If the terms of a series are non-negative, then the
associated sequence of partial sums is non-decreasing.
\begin{corollary}
  Consider the series $\ds\sum_{k=0}^\infty a_k$.  Assume the terms
  $a_k$ are non-negative.  If the sequence of partial sums $s_n = a_0
  + \cdots + a_n$ is bounded, then the series converges.
\end{corollary}
So we can show that a series of positive terms converges, provided we
can bound the sequence of partial sums.  

\subsection{Statement of the Comparison Test}

But how can we manage to do that?  One way to ensure that the sequence
of partial sums is bounded is by \textbf{comparing} the series to
another series.  Consider two series
$$
\ds\sum_{k=0}^\infty a_k \mbox{ and }
\ds\sum_{k=0}^\infty b_k.
$$
Suppose, for all $k$, that $b_k \geq a_k \geq 0$.  Then
$$
a_0 + a_1 + \cdots + a_n \leq b_0 + b_1 + \cdots + b_n.
$$
Suppose that $\ds\sum_{k=0}^\infty b_k$ converges to $L$.  Then
$$
a_0 + a_1 + \cdots + a_n \leq b_0 + b_1 + \cdots + b_n \leq L,
$$
so the sequence of partial sums $s_n = a_0 + a_1 + \cdots + a_n$ is
bounded.  But we just won the game: each term $a_k$ is nonnegative, so
the sequence of partial sums $s_n = \sum_{k=0}^n a_k$ is increasing.
Theorem~\xrefn{thm:bounded-monotonic} guarantees that the sequence
$(s_n)$ converges.

Let's summarize what just happened: if a series with positive terms
is, termwise, less than a convergent series, it converges.  We have
just proved half of the following theorem.
\begin{theorem}\label{thm:comparison-test}\index{comparison test}
Suppose that $\ds a_n$ and $\ds b_n$ are non-negative for all $n$ and
that, for some $N$, whenever $n \geq N$, we have $\ds a_n \leq b_n$.

If $\ds\sum_{n=0}^\infty b_n$ converges, so does $\ds\sum_{n=0}^\infty a_n$.

If $\ds\sum_{n=0}^\infty a_n$ diverges, so does $\ds\sum_{n=0}^\infty b_n$.
\end{theorem}
This is usually called the \defnword{Comparison Test}; we might summarize it like this:
\begin{itemize}
\item A non-negative series, overestimated by a convergent series, converges.
\item A non-negative series, underestimated by a divergent series, diverges.
\end{itemize}

\begin{warning}
  Being less than a divergent series does not help: the comparison
  test is silent in that case.

  Similarly, being larger than a convergent series does not help.
  The Comparison Test only says something when a series (with non-negative terms!) is less than a
  convergent series, or greater than a divergent series.
\end{warning}

\subsection{Applications of the Comparison Test}

Like the $n^{\nth}$ term test (Theorem~\xrefn{thm:nth-term-test}), we
can use the Comparison Test (Theorem~\xrefn{thm:comparison-test}) to
show that a series diverges.

\begin{example}
Does the series $\ds\sum_{n=2}^\infty \ds\frac{\log n}{n}$ converge?
\end{example}

\begin{solution}
Our first inclination might be to apply the $n^{\nth}$ term test, but in this case,
$$
\lim_{n \to \infty} \frac{\log n}{n} = 0,
$$
so the $n^{\nth}$ term test is silent in this case.  As far as we know
at this point, the series may diverge or converge.

Instead, we'll try the Comparison Test.  Set $a_n = \ds\frac{1}{n}$ and $b_n = \ds\frac{\log n}{n}$.  Note that whenever $n \geq 3$, we have
$$
0 \leq a_n \leq b_n,
$$
but the series $\ds\sum_{n=3}^\infty \frac{1}{n}$ diverges, and so by
the Comparison Test, the given series (which is even bigger!) must
likewise diverge.
\end{solution}

Recall that the $n^{\nth}$ term test \textit{cannot} be used to prove that a series converges; if the $n^{\nth}$ term test does not answer ``diverges!'' then the test is silent.  In wonderful contrast, the Comparison Test \textit{can} be used to show that a series converges.

\begin{example}
Does the series $\ds\sum_{n=1}^\infty \ds\frac{\sin^2 n}{2^n}$ converge?
\end{example}

\begin{solution}
Yes.  Set
$$
a_n = \frac{\sin^2 n}{2^n} \mbox{ and } b_n = \frac{1}{2^n}
$$
Note that $0 \leq a_n \leq b_n$.  But the series $\sum_{n=1}^\infty
b_n$ converges, since it is a geometric series with common ratio
$1/2$, as in Example~\xrefn{example:sum-of-half-powers}.  Therefore,
the series $\ds\sum_{n=1}^\infty a_n$ converges by the comparison
test.
\end{solution}

\subsection{Cauchy Condensation Test}

\marginnote{If you have already seen some convergence tests
  before---perhaps you have already been through Calculus Two!---you
  might be wondering why ``condensation'' is making an appearance.  It
  is perhaps less popular than other tests, but I like it.
  Pedagogically, it is nice to see that the ``trick'' in the harmonic
  series can be generalized and applied to lots of other series.  In
  particular, condensation permits the study of $p$-series without
  going through the usual route of the Integral Test.}

Remember in Section~\xrefn{section:harmonic-series} when we considered
the harmonic series?  We showed that it diverged by comparing it with
the divergent series $\ds\sum_{n=1}^\infty \frac{1}{2}$, but we
couldn't make that comparison right away---first we had to group
together the terms in a somewhat complicated seeming way.  

We can generalize that ``grouping together'' trick; this is called the
Cauchy Condensation Test.

\begin{theorem}\label{thm:cauchy-condensation}\index{Cauchy condensation test}
Suppose $(a_n)$ is a non-increasing sequence of positive numbers.  The series
$\ds\sum_{n=1}^\infty a_n$ converges if and only if the series $\ds\sum_{n=0}^\infty \left( 2^n a_{2^n} \right)$ converges.
\end{theorem}

The series $\ds\sum_{n=0}^\infty \left( 2^n a_{2^n} \right)$ is often
called the \defnword{condensed}\index{condensed
  series}\index{series!condensed} series associated to the series
$\ds\sum_{n=1}^\infty a_n$.

\begin{proof}
  Let's suppose that $\ds\sum_{n=0}^\infty \left( 2^n a_{2^n} \right)$
  converges; the goal then is to show that $\ds\sum_{n=1}^\infty a_n$
  also converges.

  Since the sequence $(a_n)$ is decreasing, we have that
  \begin{align*}
  a_2 + a_3 &< a_2 + a_2 \\
  a_4 + a_5 + a_6 + a_7 &< a_4 + a_4 + a_4 + a_4 \\
  a_8 + \cdots + a_{15} &< 8 \, a_8 \\
  &\vdots \\
  a_{2^n} + \cdots + a_{2^{n+1} - 1} &< 2^n \, a_{2^n}.
  \end{align*}
  Therefore,
  $$
  \sum_{n=1}^{2^k - 1} a_n \leq \sum_{n=0}^{k-1} 2^n \, a_{2^n}.
  $$
  As a result, the sequence of partial sums $s_k = \sum_{n=1}^k a_n$ is bounded above by $\sum_{n=0}^\infty 2^n \, a_{2^n}$.  Moreover, the sequence of partial sums $(s_k)$ is increasing.  Therefore, by the Monotone Convergence Theorem, the series $(s_k)$ converges.

  On the other hand, suppose that $\ds\sum_{n=1}^\infty a_n$ converges; let's show that $\ds\sum_{n=0}^\infty \left( 2^n a_{2^n} \right)$
  also converges.  Once we have done so, we will have shown that  $\ds\sum_{n=1}^\infty a_n$ converges if and only if $\ds\sum_{n=0}^\infty \left( 2^n a_{2^n} \right)$ converges.

  Since the sequence $(a_n)$ is decreasing, we have that
  \begin{align*}
  a_1 + a_2 &< a_1 + a_1 \\
  a_2 + 3\,a_4 &< a_2 + a_2 + a_3 + a_3 \\
  a_4 + 7 \, a_8 &< a_4 + a_4 + a_5 + a_5 + a_6 + a_6 + a_7 + a_7 \\
  a_8 + 15 \, a_{16} &< 2(a_8 + \cdots + a_{15}) \\
  &\vdots \\
  a_{2^n} + (2^n - 1) \, a_{2^{n+1}} &< 2(a_{2^{n}} + \cdots + a_{2^{n+1} - 1}).
  \end{align*}
  So the sequence of partial sums for the series $\ds\sum_{n=0}^k \left( 2^n a_{2^n} \right)$ are bounded above by $2 \cdot \sum_{n=1}^\infty a_n$.  Moreover, that sequence of partial sums is increasing, and therefore, by the Monotone Convergence Theorem, the series $\ds\sum_{n=0}^\infty \left( 2^n a_{2^n} \right)$ converges---which is what we wanted to show.  
\end{proof}

What we've shown is a bit stronger than simply that the original series and the condensed series share the same fate---converging or diverging together.  In fact, we have an estimate on the value of the original series, in terms of the value of the condensed series.  We have shown that
$$
\sum_{n=1}^\infty a_n \leq \ds\sum_{n=0}^k \left( 2^n a_{2^n} \right) \leq 2 \cdot \sum_{n=1}^\infty a_n.
$$

\subsection{Examples of condensation}

\begin{example}
Does the series $\ds\sum_{n=1}^\infty \ds\frac{1}{n^2}$ converge?
\end{example}

This is \textit{not} a geometric series: we already know that $\ds\sum_{n=1}^\infty \ds\frac{1}{2^n}$ converges, but this is asking about something very different, namely $\ds\sum_{n=1}^\infty \ds\frac{1}{n^2}$.

\begin{solution}
To get some intuition for what is going on, let's do some numerical calculations.
\begin{align*}
\sum_{n=1}^{10} \frac{1}{n^2} &= \frac{1}{1^2} + \frac{1}{2^2} + \frac{1}{3^2} + \cdots + \frac{1}{10^2} \\
&= \frac{1968329}{1270080} \approx 1.5498,
\end{align*}
or going out a bit farther,
\begin{align*}
\sum_{n=1}^{100} \frac{1}{n^2} = \frac{1}{1^2} + \cdots + \frac{1}{100^2} \approx 1.6350 \mbox{ and } \\
\sum_{n=1}^{1000} \frac{1}{n^2} = \frac{1}{1^2} + \cdots + \frac{1}{1000^2} \approx 1.6439. \\
\end{align*}
From this numerical evidence, it certainly \textit{looks} like this
series converges.  And indeed, it does---quite surprisingly,
$$
\sum_{n=1}^{\infty} \frac{1}{n^2} = \frac{\pi^{2}}{6}.
$$
This is the so-called
\href{http://en.wikipedia.org/wiki/Basel_problem}{Basel
  problem}\index{Basel problem}.

We do not yet have the tools necessarily to show that the value of the
series is $\pi^2/6$, but do we have the tools needed to show that the
series converges.  By condensation, it suffices to show that
$\ds\sum_{n=1}^\infty \ds\frac{2^n}{\left( 2^n \right)^2}$ converges.
But
$$
\ds\sum_{n=1}^\infty \ds\frac{2^n}{\left( 2^n \right)^2} = 
\ds\sum_{n=1}^\infty \ds\frac{1}{\left( 2^{n} \right)} = 1,
$$
and since the ``condensed'' series converges, so too must the original series converge.
\end{solution}

\begin{example} Does $\ds\sum_{n=2}^\infty {|\sin n|\over n^2}$ converge?
\end{example}

\begin{solution}
  We can't apply Cauchy condensation here, because the terms of this series
  are not decreasing.  But we can apply the Comparison Test.  Moments ago, we
  saw that $\ds\sum_{n=1}^\infty \ds\frac{1}{n^2}$ converges, and
$$ {|\sin n|\over n^2}\le {1\over n^2},$$
because $|\sin n|\le 1$.  The partial sums are
non-decreasing and bounded above by $\ds \sum_{n=1}^\infty 1/n^2=L$, so the series
converges. 
\end{solution}
\label{example:absolute sine over n squared}

\subsection{Convergence of $p$-series}

Let us consider the series $\ds\sum_{n=1}^\infty \ds\frac{1}{n^p}$.  Such a series is called a \defnword{$\mathbf{p}$-series}\index{$p$-series}.
Does a $p$-series converge?  Diverge?  It depends on $p$.
\marginnote{If we think of this as a function of $p$, then we have the \defnword{Riemann zeta function}, that is,
$$\zeta(p) = \ds\sum_{n=1}^\infty \ds\frac{1}{n^p}.$$
The Riemann zeta function is quite important: it plays a key role in number theory via the \href{http://en.wikipedia.org/wiki/Riemann_hypothesis}{\defnword{Riemann hypothesis}} and also has applications in physics.  Something that connects the physical world to number theory must be pretty incredible.}

\begin{example}
Let $p \leq 1$.  Does the series $\ds\sum_{n=1}^\infty \ds\frac{1}{n^p}$ converge?
\end{example}

\begin{solution}
  When $p=1$, this series is the harmonic series we already proved to diverge
  in Section~\xrefn{section:harmonic-series}. 

  But more generally, the series $\ds\sum_{n=1}^\infty \ds\frac{1}{n^p}$ diverges whenever
  $p \leq 1$.  We will show this by comparing to a harmonic series.
  Since $p \leq 1$, then $n^p \leq n$, and so
$$
\frac{1}{n^p} \geq \frac{1}{n}.
$$
But the harmonic series $\ds\sum_{n=1}^\infty \ds\frac{1}{n}$
diverges, and so by comparison, the series $\ds\sum_{n=1}^\infty
\ds\frac{1}{n^p}$ diverges.
\end{solution}

\begin{example}
Let $p > 1$.  Does the series $\ds\sum_{n=1}^\infty \ds\frac{1}{n^p}$ converge?
\end{example}

\begin{solution}
  It converges.  For this, we use Cauchy condensation: consider the ``condensed'' series
$$
\ds\sum_{n=1}^\infty 2^n \cdot \ds\frac{1}{\left( 2^n \right)^p}.
$$
But this series simplifies to
$$
\ds\sum_{n=1}^\infty 2^n \cdot \ds\frac{1}{\left( 2^n \right)^p} =
\ds\sum_{n=1}^\infty  \ds\frac{1}{\left( 2^{p-1} \right)^n},
$$
which converges.
\end{solution}

