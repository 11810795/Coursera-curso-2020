Consider the following sum:
$${1\over2}+{1\over4}+{1\over8}+{1\over16}+\cdots+{1\over2^i}+\cdots$$
The dots at the end indicate that the sum goes on forever. Does this
make sense? Can we assign a numerical value to an infinite sum? While
at first it may seem difficult or impossible, we have certainly done
something similar when we talked about one quantity getting ``closer
and closer'' to a fixed quantity. Here we could ask whether, as we add
more and more terms, the sum gets closer and closer to some fixed
value. That is, look at
%$$\eqalign{
%{1\over2}&={1\over2} \\
%{3\over4}&={1\over2}+{1\over4} \\
%{7\over8}&={1\over2}+{1\over4}+{1\over8} \\
%{15\over16}&={1\over2}+{1\over4}+{1\over8}+{1\over16} \\
%}$$ BADBAD
and so on, and ask whether these values have a limit. It seems pretty
clear that they do, namely $1$. In fact, as we will see, it's not hard
to show that 
$${1\over2}+{1\over4}+{1\over8}+{1\over16}+\cdots+{1\over2^i}=
{2^i-1\over2^i}=1-{1\over2^i}$$
and then
$$\lim_{i\to\infty} 1-{1\over2^i}=1-0=1.$$
There is one place that you have long accepted this notion of infinite
sum without really thinking of it as a sum:
$$0.3333\bar3 =
{3\over10}+{3\over100}+{3\over1000}+{3\over10000}+\cdots=
{1\over3},$$
for example, or
$$3.14159\ldots = 3+{1\over10}+{4\over100}+{1\over1000}+{5\over10000}+
{9\over100000}+\cdots = \pi.$$
Our first task, then,  to investigate infinite sums, called 
{\dfont series\index{series}\/}, is to investigate limits of {\dfont
  sequences\index{sequence}\/} of numbers. That is, we officially
call
$$\sum_{i=1}^\infty {1\over2^i}=
{1\over2}+{1\over4}+{1\over8}+{1\over16}+\cdots+{1\over2^i}+\cdots$$
a series, while
$${1\over2},{3\over4},{7\over8},{15\over16},\ldots,{2^i-1\over2^i},\ldots$$
is a sequence, and
$$\sum_{i=1}^\infty {1\over2^i}=\lim_{i\to\infty} {2^i-1\over2^i},$$
that is, the value of a series is the limit of a particular sequence.


