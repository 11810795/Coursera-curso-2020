\section{Definitions}
\nobreak
At first, we studied the geometric series
$$
  \sum_{n=0}^\infty \left(\frac{1}{2}\right)^n,
$$
but then we replaced $\frac{1}{2}$ with a variable, so we were able to
analyze all geometric series simultaneously.  In the end, we
discovered that
$$
  \sum_{n=0}^\infty kx^n = {k\over 1-x} \mbox{ if $|x| < 1$,}
$$
\marginnote{You may recall that we used $r$ instead of $x$; I think $r$ was a good choice back then, since $r$ evokes ``ratio'' and stood for the common ratio between the terms in a geometric series.  Now we use $x$, which I think is a good choice now.  Why the change?  To me, $x$ suggests ``independent variable of a function'' much more generally than just a parameter in a geometric series, and we will be thinking very broadly about how quite general functions can be represented as a power series.  One may regard power series as a natural generalization of polynomials, and you may already be comfortable thinking of $x$ as the variable in a polynomial.}
and that the series diverges when $|x|\ge 1$. At the time,
we thought of $x$ as an unspecified constant, but we could just as
well think of it as a variable.  A number which depends on another
number is just a function, so we may write
$$
  f(x) = \sum_{n=0}^\infty kx^n
$$
and then observe that it is also the case that $f(x) = k/(1-x)$, as
long as $|x|<1$. While $k/(1-x)$ is a reasonably easy function to deal
with, the more complicated $\sum kx^n$ does have its attractions: it
appears to be an infinite version of one of the simplest sorts of
functions---a polynomial. Do other functions have representations as
series? Is there an advantage to viewing them in this way?

Usually the coefficients aren't all the same in a polynomial; the
geometric series is somewhat unusual in that all the coefficients of
the powers of $x$ are the same, namely $k$.  We will need to allow more
general coefficients if we are to get anything other than the
geometric series.

\begin{definition}
  \label{defn:power-series}
  Let $(a_n)$ be a sequence of real numbers starting with $a_0$.  Then
  the \defnword{power series}\index{power series} associated to $(a_n)$ is
  $$
  \sum_{n=0}^\infty a_n \, x^n.
  $$ 
\end{definition}

Sometimes people are confused when considering, say,
$$
\sum_{n=0}^\infty (\sin x) \cdot x^n.
$$
Despite appearances, we will not be regarding such series as ``power
series.''  Since $(a_n)$ is a sequence of real numbers, the
coefficients $a_n$ cannot depend on $x$.

\section{Convergence of power series}
\label{section:convergence-of-power-series}

\marginnote{In Example~\xrefn{example:power-series-xn-over-n}, note
  that we are no longer considering convergence of a single series;
  instead, by regarding $x$ as a parameter, we are considering
  convergence for a whole family of series---namely all those of the
  form $\sum_{n=1}^\infty {x^n/n}$.  It often happens in mathematics
  that it is wiser to study a family of objects simultaneously than to
  study a single object in isolation.}

\begin{example}
\label{example:power-series-xn-over-n}
Consider the power series
$$f(x) = \sum_{n=1}^\infty {x^n\over n}.$$
For which $x$ does this converge?
\end{example}

Another way to think of this question is this: we are being asked to
determine the domain of $f$.

\begin{solution}
  We can investigate the convergence of this series using the ratio
  test.  The difficulty now is that, instead of a single limit, we
  must now compute a limit than involves a variable $x$ which is
  \textit{not} the variable with respect to which we are taking a
  limit.  

  Another issue is that $x$ may be negative, so we will actually check
  absolute convergence; in other words, we are actually considering
  $$
  f(x) = \sum_{n=1}^\infty \left| \frac{x^n}{n} \right|.
  $$
  Bravely,
  \begin{align*}
    L &= \lim_{n\to\infty} \frac{\frac{|x|^{n+1}}{n+1}}{\frac{|x|^{n}}{n}} \\
      &= \lim_{n\to\infty} \frac{|x|^{n+1}}{n+1} \cdot \frac{1}{\frac{|x|^{n}}{n}} \\
      &= \lim_{n\to\infty} \frac{|x|^{n+1}}{n+1} \cdot \frac{n}{|x|^{n}} \\
      &= \lim_{n\to\infty} \frac{|x|^{n+1} \cdot n}{(n+1) \cdot |x|^{n}} \\
      &= \lim_{n\to\infty} |x| \frac{n}{n+1} = |x| \cdot \lim_{n\to\infty} \frac{n}{n+1} = |x|.
  \end{align*}
  So when $L = |x| < 1$, the ratio test says that the series converges
  absolutely.  When $|x| > 1$, the series does not converge
  absolutely---in fact, when $|x| > 1$, we have that
  $$
  \lim_{n \to \infty} \frac{x^n}{n} \neq 0
  $$
  and so the series diverges.

  So when $|x|<1$ the series converges and when $|x|>1$ it diverges,
  which leaves only two values in doubt. When $x=1$, the series is the
  harmonic series and diverges; when $x=-1$, it is the alternating
  harmonic series (rather, the negative of the ``usual'' alternating
  harmonic series) and converges. 

  In other words, we may regard $f$ as a function with domain $[-1,1)$.
\end{solution}

We analyzed this power series by invoking the ratio test, but that was no
accident.  We will see that the ratio test applied to a power series
will always have the same nice form.  Feeling confident from our
display of bravery before the preceding example, let's attack the
general case of a power series
$$
f(x) = \sum_{n=0}^\infty a_n\, x^n.
$$
When does this series converge absolutely?  Applying the ratio test again, we find
\begin{align*}
  L &= \lim_{n\to\infty} \frac{|a_{n+1}| \cdot |x|^{n+1}}{|a_n| \cdot |x|^{n}} \\
  &= \lim_{n\to\infty} |x| \cdot \frac{|a_{n+1}|}{|a_n|} \\
  &= |x| \cdot \lim_{n\to\infty} \frac{|a_{n+1}|}{|a_n|}.
\end{align*}
So the series converges absolutely whenever $L < 1$, but in this case, $L$ depends in a rather uncomplicated way on $|x|$.  The whole story is controlled by a limit that \textit{does not} depend on $x$, namely
$$
  1/R = \lim_{n\to\infty} \frac{|a_{n+1}|}{|a_n|},
$$
which you see I have presciently related to a hitherto unmentioned
variable, $R$.  So the series converges absolutely whenever $|x| / R <
1$, meaning the series converges absolutely whenever $|x| < R$.  When
$|x| > R$, the series diverges\sidenote{This is a consequent of the
  ratio test, but perhaps we haven't emphasized it enough; if $L > 1$
  in the ratio test, then the limit of the terms is not zero, so not
  only does the series not converge absolutely---the original series
  diverges, too, by the limit test.}.

Once again, only the two values $x=\pm R$ require further
investigation.  In any case, if we begin with a sequence $(a_n)$ for which
$$
  1/R = \lim_{n\to\infty} \frac{|a_{n+1}|}{|a_n|},
$$
the associated power series $\sum_{n=0}^\infty a_n x^n$ will converge
for values of $x$ in the interval $(-R,R)$, and it may even converge
(or not, depending on the specific situation) at $x = R$ or at $x =
-R$, so sometimes the interval on which the series makes sense will be
$[-R,R)$ or $(-R,R]$ or even $[-R,R]$.

\begin{definition}
  \label{definition:interval-of-convergence}
  The set of values of $x$ for which the series $\sum_{n=0}^\infty a_n \, x^n$ converges is the
  \defnword{interval of convergence}\index{interval of
    convergence}\index{series!interval of convergence}.
\end{definition}

As a consequence of the above discussion, the interval of convergence
of a power series always have a nice form.

\begin{theorem}
  \label{thm:convergence-set-is-interval}
  For a power series, the interval of convergence is, in fact, an
  interval.  It has the form $(R,-R)$ or $[-R,R)$ or $(-R,R]$ or
  $[-R,R]$.  In short, it is centered around $0$.
\end{theorem}

Because the interval of convergence is, indeed, centered around $0$,
and because we often don't care about what happens at the endpoint, it
is often convenient to just describe $R$.

\begin{definition}
  \label{definition:radius-of-convergence}
  In the interval of convergence of a power series, the value $R$ is called
  the \defnword{radius of convergence}\index{radius of convergence}\index{series!radius of convergence} of the series.
\end{definition}

Two special cases deserve mention.  If
$$
\lim_{n\to\infty} \frac{|a_{n+1}|}{|a_n|} = 0,
$$
then we might say ``$R = \infty$'' since no matter what $x$ is, the product
$$
|x| \lim_{n\to\infty} \frac{|a_{n+1}|}{|a_n|} = 0,
$$
and therefore for all values of $x$, the power series
$\sum_{n=0}^\infty a_n \, x^n$ converges.  So the radius of
convergence, $R$, is infinite.

The opposite may happen as well.  It may happen that
$$
\lim_{n\to\infty} \frac{|a_{n+1}|}{|a_n|} = \infty,
$$
in which case the only way that 
$$
|x| \lim_{n\to\infty} \frac{|a_{n+1}|}{|a_n|} < 1
$$
is when $|x| = 0$.  The interval of convergence is just the single
point $\{0\}$.  And in this case, we say $R = 0$.

\begin{warning}
  People often confuse ``radius of convergence'' with ``interval of
  convergence.''  For starters, the radius of convergence is a single
  number, while the interval of convergence is, well, an interval
  (though it was defined as just a set---the fact that the set where
  the series converges \textit{is} an interval was a theorem and was
  certainly not obvious---perhaps the first of many lovely surprises
  with power series).

  There is another difference between the radius of convergence and
  the interval of convergence; it is not simply that they are the same
  information in different packages.  The radius of convergence cannot
  distinguish between, say, $(-R,R)$ versus $[-R,R]$.  The interval of
  convergence contains the extra information about what is happening
  at the endpoints.  So a homework question which asks you to find the
  radius $R$ is much easier than a homework question asking you to
  find the interval.
\end{warning}

\section{Power series centered elsewhere}
\label{section:power-series-centered-around-a}

If we are speaking of the ``radius'' of convergence, you might wonder
about the ``center'' of convergence.  Thus far, our intervals of
convergence have been centered around the origin, but by changing the
series, we can move the interval of convergence.  Let's see how.

Consider once again the geometric series,
$$
f(x) = \sum_{n=0}^\infty x^n={1\over 1-x} \mbox{ for $x \in (-1,1)$.}
$$
Whatever benefits there might be in using the series form of this
function are only available to us when $x$ is between $-1$ and
$1$.  We can address this shortcoming by modifying the power
series slightly.

\begin{example}
\label{example:one-over-one-minus-x-on-different-interval}
Find a series representation for $\frac{1}{1-x}$ valid on the interval $(-5,1)$.
\end{example}

\begin{solution}
Consider that the series
$$
  \sum_{n=0}^\infty {(x+2)^n\over 3^n}=
  \sum_{n=0}^\infty \left({x+2\over 3}\right)^n={1\over 1-{x+2\over 3}}=
  {3\over 1-x},
$$
because this is just a geometric series with $x$ replaced by
$(x+2)/3$. Multiplying both sides by $1/3$
gives %\pagerdef{page:alt series for 1 over 1-x}
$$\sum_{n=0}^\infty {(x+2)^n\over 3^{n+1}}={1\over 1-x},$$
the same function as before. For what values of $x$ does this series
converge? Since it is a geometric series, we know that it converges
when 
\begin{align*}
  |x+2|/3&<1 \\
  |x+2|&<3 \\
  -3 < x+2 &< 3 \\
  -5<x&<1. \\
\end{align*}
So we have a series representation for $1/(1-x)$ that works on a
larger interval than before, at the expense of a somewhat more
complicated series. The endpoints of the interval of convergence now
are $-5$ and $1$, but note that those two endpoints can be described
as $-2\pm3$. We say that $3$ is the radius of convergence, and we now
say that the series is centered at $-2$.
\end{solution}

Let's capture this in a definition.

\marginnote{It is worth contrasting
  Definition~\xrefn{defn:power-series-centered-at-c} with
  Definition~\xrefn{defn:power-series}, our original description of
  power series.  In particular, $\sum_{n=0}^\infty a_n \, x^n$ is a
  power series centered at zero.}

\begin{definition} 
  \label{defn:power-series-centered-at-c}
  Let $(a_n)$ be a sequence of real numbers starting with $a_0$.  Then
  the \defnword{power series} centered at $c$ and associated to $(a_n)$ is the series
  $$
  \sum_{n=0}^\infty a_n \, (x-c)^n.
  $$ 
\end{definition}

In Example~\xrefn{example:one-over-one-minus-x-on-different-interval},
we formed a series that involved $(x+2)^n$, meaning that in that case
$c = -2$.

You are now in a position to try your hand at finding the radius of
convergence and sometimes even the interval of convergence for some
power series.  I encourage you to cook up your own power series, and
then see what you can say about the radius and interval of
convergence.

% BADBAD: should discuss taylor(1/(1-e^(I*x)),x,0,10)

\begin{exercises}

Find the radius and interval of convergence for each series.  In
exercises~\xrefn{exer:no-endpoints-one} and~\xrefn{exer:no-endpoints-two},
do not attempt to determine whether the endpoints are in the
interval of convergence.

\begin{exercise} $\ds\sum_{n=0}^\infty n x^n$
\begin{answer} $R=1$, $I=(-1,1)$
\end{answer}\end{exercise}

\begin{exercise} $\ds\sum_{n=0}^\infty {x^n\over n!}$
\begin{answer} $R=\infty$, $I=(-\infty,\infty)$
\end{answer}\end{exercise}

\begin{exercise} 
\relax\label{exer:no-endpoints-one}
$\ds\sum_{n=1}^\infty {n!\over n^n}x^n$
\begin{answer} $R=e$, $I=(-e,e)$
\end{answer}\end{exercise}

\begin{exercise} 
\relax\label{exer:no-endpoints-two}
$\ds\sum_{n=1}^\infty {n!\over n^n}(x-2)^n$
\begin{answer} $R=e$, $I=(2-e,2+e)$
\end{answer}\end{exercise}

\begin{exercise}
\relax\label{exer:radius-of-convergence-infinity}
 $\ds\sum_{n=1}^\infty {(n!)^2\over n^n}(x-2)^n$
\begin{answer} $R=0$, converges only when $x=2$
\end{answer}\end{exercise}

\begin{exercise} $\ds\sum_{n=1}^\infty {(x+5)^n\over n(n+1)}$
\begin{answer} $R=1$, $I=[-6,-4]$
\end{answer}\end{exercise}

\begin{exercise} Find a power series with radius of convergence $0$.
\begin{answer} There are many choices---for instance, see Exercise~\xrefn{exer:radius-of-convergence-infinity}---but $\sum_{n=0}^\infty n! \cdot x^n$ works.
\end{answer}\end{exercise}

\end{exercises}

