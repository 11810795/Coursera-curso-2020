
\nobreak A bounded, monotonic sequence necessarily converges
(Theorem~\xrefn{thm:bounded-monotonic}).  How does this fact about
sequences relate to series?  When is the sequence of partial sums
monotonic?  If the terms of a series are non-negative, then the
associated sequence of partial sums is non-decreasing.
\begin{corollary}
  Consider the series $\ds\sum_{k=0}^\infty a_k$.  Assume the terms
  $a_k$ are non-negative.  If the sequence of partial sums $s_n = a_0
  + \cdots + a_n$ is bounded, then the series converges.
\end{corollary}
So we can show that a series of positive terms converges, provided we
can bound the sequence of partial sums.  

\subsection{Statement of comparison test}

But how can we manage to do that?  One way to ensure that the sequence
of partial sums is bounded is by \textbf{comparing} the series to
another series.  Consider two series
$$
\ds\sum_{k=0}^\infty a_k \mbox{ and }
\ds\sum_{k=0}^\infty b_k.
$$
Suppose, for all $k$, that $b_k \geq a_k > 0$.  Then
$$
a_0 + a_1 + \cdots + a_n \leq b_0 + b_1 + \cdots + b_n.
$$
Suppose that $\ds\sum_{k=0}^\infty b_k$ converges to $L$.  Then
$$
a_0 + a_1 + \cdots + a_n \leq b_0 + b_1 + \cdots + b_n \leq L,
$$
so the sequence $s_n = a_0 + a_1 + \cdots + a_n$ is bounded.  But we
just won the game: if a series with positive terms is, termwise, less
than a convergent series, it converges.  We have just proved half of
the following theorem.
\begin{theorem}\label{thm:comparison-test}\index{comparison test}
Suppose that $\ds a_n$ and $\ds b_n$ are non-negative for all $n$ and
that, for some $N$, whenever $n \geq N$, we have $\ds a_n \leq b_n$.

If $\ds\sum_{n=0}^\infty b_n$ converges, so does $\ds\sum_{n=0}^\infty a_n$.

If $\ds\sum_{n=0}^\infty a_n$ diverges, so does $\ds\sum_{n=0}^\infty b_n$.
\end{theorem}
This is usually called the \defnword{comparison test}; we might summarize it like this:
\begin{quote}
  A non-negative series, overestimated by a convergent series, converges. \\
  A non-negative series, underestimated by a divergent series, diverges.
\end{quote}

\begin{warning}
  Being less than a divergent series does not help: the comparison
  test is silent in that case.

  Similarly, because larger than a convergent series does not help.
  The comparison test only says something when a series is less than a
  convergent series, or greater than a divergent series.
\end{warning}

\subsection{Examples of comparison tests}
%BADBAD

\subsection{Cauchy condensation}
%BADBAD

\subsection{Examples of condensation}

\begin{example}
Does the series $\ds\sum_{n=1}^\infty \ds\frac{1}{n^2}$ converge?
\end{example}

\begin{solution}
To get some intuition for what is going on, let's do some numerical calculations.
\begin{align*}
\sum_{n=1}^{10} \frac{1}{n^2} &= \frac{1}{1^2} + \frac{1}{2^2} + \frac{1}{3^2} + \cdots + \frac{1}{10^2} \\
&= \frac{1968329}{1270080} \approx 1.5498,
\end{align*}
or going out a bit farther,
\begin{align*}
\sum_{n=1}^{100} \frac{1}{n^2} = \frac{1}{1^2} + \cdots + \frac{1}{100^2} \approx 1.6350 \mbox{ and } \\
\sum_{n=1}^{1000} \frac{1}{n^2} = \frac{1}{1^2} + \cdots + \frac{1}{1000^2} \approx 1.6439. \\
\end{align*}
From this numerical evidence, it certainly \textit{looks} like this
series converges.  And indeed, it does---quite surprisingly,
$$
\sum_{n=1}^{\infty} \frac{1}{n^2} = \frac{\pi^{2}}{6}.
$$
This is the so-called
\href{http://en.wikipedia.org/wiki/Basel_problem}{Basel
  problem}\index{Basel problem}.

We do not yet have the tools necessarily to show that the value of the
series is $\pi^2/6$, but do we have the tools needed to show that the
series converges.  By condensation, it suffices to show that
$\ds\sum_{n=1}^\infty \ds\frac{2^n}{\left( 2^n \right)^2}$ converges.
But
$$
\ds\sum_{n=1}^\infty \ds\frac{2^n}{\left( 2^n \right)^2} = 
\ds\sum_{n=1}^\infty \ds\frac{1}{\left( 2^{n} \right)} = 1,
$$
and since the ``condensed'' series converges, so too must the original series converge.
\end{solution}

\begin{example} Does $\ds\sum_{n=2}^\infty {|\sin n|\over n^2}$ converge?
\end{example}

\begin{solution}
  We can't apply a condensation here, because the terms of this series
  are not decreasing.  But we can apply a comparison test.  Moments ago, we
  saw that $\ds\sum_{n=1}^\infty \ds\frac{1}{n^2}$ converges, and
$$ {|\sin n|\over n^2}\le {1\over n^2},$$
because $|\sin n|\le 1$.  The partial sums are
non-decreasing and bounded above by $\ds \sum_{n=1}^\infty 1/n^2=L$, so the series
converges. 
\end{solution}
\label{example:absolute sine over n squared}

\subsection{Convergence of $p$-series}

Let us consider the series $\ds\sum_{n=1}^\infty \ds\frac{1}{n^p}$.  Such a series is called a \defnword{$\mathbf{p}$-series}\index{$p$-series}.
Does a $p$-series converge?  Diverge?  It depends on $p$.
\marginnote{If we think of this as a function of $p$, then we have the \defnword{Riemann zeta function}, that is,
$$\zeta(p) = \ds\sum_{n=1}^\infty \ds\frac{1}{n^p}.$$
The Riemann zeta function is quite important: it plays a key role in number theory via the \href{http://en.wikipedia.org/wiki/Riemann_hypothesis}{Riemann hypothesis}\defnword{Riemann hypothesis} and also has applications in physics.  Something that connects the physical world to number theory must be pretty incredible.}

\begin{example}
Let $p \leq 1$.  Does the series $\ds\sum_{n=1}^\infty \ds\frac{1}{n^p}$ converge?
\end{example}

When $p=1$, this series is the harmonic series we already proved to diverge
in Section~\xrefn{section:harmonic-series}.

\begin{solution}
  The series $\ds\sum_{n=1}^\infty \ds\frac{1}{n^p}$ diverges whenever
  $p \geq 1$.  We will show this by comparing to a harmonic series.
  Since $p \leq 1$, then $n^p \leq n$, and so
$$
\frac{1}{n^p} \geq \frac{1}{n}.
$$
But the harmonic series $\ds\sum_{n=1}^\infty \ds\frac{1}{n}$
diverges, and so by comparison, the series $\ds\sum_{n=1}^\infty
\ds\frac{1}{n^p}$ diverges.
\end{solution}

\begin{example}
Let $p \geq 1$.  Does the series $\ds\sum_{n=1}^\infty \ds\frac{1}{n^p}$ converge?
\end{example}

\begin{solution}
  It converges.  For this, we use Cauchy condensation: consider the ``condensed'' series
$$
\ds\sum_{n=1}^\infty 2^n \cdot \ds\frac{1}{\left( 2^n \right)^p}.
$$
But this series simplifies to
$$
\ds\sum_{n=1}^\infty 2^n \cdot \ds\frac{1}{\left( 2^n \right)^p} =
\ds\sum_{n=1}^\infty  \ds\frac{1}{\left( 2^{p-1} \right)^n},
$$
which converges.
\end{solution}

%%%%%%%%%%%%%%%%%%%%%%%%%%%%%%%%%%%%%%%%%%%%%%%%%%%%%%%%%%%%%%%%
%BADBAD
\begin{example} Does $\ds\sum_{n=2}^\infty {1\over n^2\ln n}$ converge?
\end{example}

\begin{solution}
The obvious first approach, based on what we know, is the integral test.
Unfortunately, we can't compute the required antiderivative. But
looking at the series, it would appear that it must converge, because
the terms we are adding are smaller than the terms of a $p$-series,
that is,
$${1\over n^2\ln n}<{1\over n^2},$$
when $n\ge3$. Since adding up the terms $\ds 1/n^2$ doesn't get ``too
big'', the new series ``should'' also converge. Let's make this more
precise.

The series $\ds\sum_{n=2}^\infty {1\over n^2\ln n}$ converges if and
only if $\ds\sum_{n=3}^\infty {1\over n^2\ln n}$ converges---all we've
done is dropped the initial term. We know that 
$\ds\sum_{n=3}^\infty {1\over n^2}$ converges. Looking at two typical
partial sums:
$$
  s_n={1\over 3^2\ln 3}+{1\over 4^2\ln 4}+{1\over 5^2\ln 5}+\cdots+
  {1\over n^2\ln n} < {1\over 3^2}+{1\over 4^2}+
  {1\over 5^2}+\cdots+{1\over n^2}=t_n.
$$
Since the $p$-series converges, say to $L$, and since the terms are positive,
$\ds t_n<L$. Since the terms of the new series are positive, the $\ds s_n$
form an increasing sequence and $\ds s_n<t_n<L$ for all $n$. Hence the
sequence $\ds \{s_n\}$ is bounded and so converges.
\end{solution}

Sometimes, even when the integral test applies, comparison to a known
series is easier, so it's generally a good idea to think about doing a
comparison before doing the integral test.

Like the integral test, the comparison test can be used to show both
convergence and divergence. In the case of the integral test, a single
calculation will confirm whichever is the case. To use the comparison
test we must first have a good idea as to convergence or divergence
and pick the sequence for comparison accordingly.

\begin{example} Does $\ds\sum_{n=2}^\infty {1\over\sqrt{n^2-3}}$ converge?
\ssk
We observe that the $-3$ should have little effect compared to the
$\ds n^2$ inside the square root, and therefore guess that the terms are
enough like $\ds 1/\sqrt{n^2}=1/n$ that the series should diverge. We
attempt to show this by comparison to the harmonic series. We note
that 
$${1\over\sqrt{n^2-3}} > {1\over\sqrt{n^2}} = {1\over n},$$
so that
$$
  s_n={1\over\sqrt{2^2-3}}+{1\over\sqrt{3^2-3}}+\cdots+
  {1\over\sqrt{n^2-3}} > {1\over 2} + {1\over3}+\cdots+{1\over n}=t_n,
$$
where $\ds t_n$ is 1 less than the corresponding partial sum of the
harmonic series (because we start at $n=2$ instead of $n=1$). Since
$\ds\lim_{n\to\infty}t_n=\infty$, $\ds\lim_{n\to\infty}s_n=\infty$ as
well.
\end{example}

So the general approach is this: If you believe that a new series is
convergent, attempt to find a convergent series whose terms are
larger than the terms of the new series; if you believe that a new
series is divergent, attempt to find a divergent series whose terms
are smaller than the terms of the new series.

\begin{example} Does $\ds\sum_{n=1}^\infty {1\over\sqrt{n^2+3}}$ converge?
\ssk
Just as in the last example, we guess that this is very much like the
harmonic series and so diverges. Unfortunately,
$${1\over\sqrt{n^2+3}} < {1\over n},$$
so we can't compare the series directly to the harmonic series.
A little thought leads us to
$${1\over\sqrt{n^2+3}} > {1\over\sqrt{n^2+3n^2}} = {1\over2n},$$ so if
$\sum 1/(2n)$ diverges then the given series diverges. But since $\sum
1/(2n)=(1/2)\sum 1/n$, theorem~\xrefn{thm:series are linear} implies
that it does indeed diverge.
\end{example}





%%%%%%%%%%%%%%%%%%%%%%%%%%%%%%%%%%%%%%%%%%%%%%%%%%%%%%%%%%%%%%%%
\begin{exercises}

Determine whether the series converge or diverge.

\twocol

\begin{exercise} $\ds\sum_{n=1}^\infty {1\over 2n^2+3n+5} $
\begin{answer} converges
\end{answer}\end{exercise}

\begin{exercise} $\ds\sum_{n=2}^\infty {1\over 2n^2+3n-5} $
\begin{answer}  converges
\end{answer}\end{exercise}

\begin{exercise} $\ds\sum_{n=1}^\infty {1\over 2n^2-3n-5} $
\begin{answer}  converges
\end{answer}\end{exercise}

\begin{exercise} $\ds\sum_{n=1}^\infty {3n+4\over 2n^2+3n+5} $
\begin{answer} diverges
\end{answer}\end{exercise}

\begin{exercise} $\ds\sum_{n=1}^\infty {3n^2+4\over 2n^2+3n+5} $
\begin{answer} diverges
\end{answer}\end{exercise}

\begin{exercise} $\ds\sum_{n=1}^\infty {\ln n\over n}$
\begin{answer} diverges
\end{answer}\end{exercise}

\begin{exercise} $\ds\sum_{n=1}^\infty {\ln n\over n^3}$
\begin{answer} converges
\end{answer}\end{exercise}

\begin{exercise} $\ds\sum_{n=2}^\infty {1\over \ln n}$
\begin{answer} diverges
\end{answer}\end{exercise}

\begin{exercise} $\ds\sum_{n=1}^\infty {3^n\over 2^n+5^n}$
\begin{answer} converges
\end{answer}\end{exercise}

\begin{exercise} $\ds\sum_{n=1}^\infty {3^n\over 2^n+3^n}$
\begin{answer} diverges
\end{answer}\end{exercise}

\endtwocol

\end{exercises}

